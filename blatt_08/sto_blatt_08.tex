\documentclass[a4paper]{scrartcl}

% font/encoding packages
\usepackage[utf8]{inputenc}
\usepackage[T1]{fontenc}
\usepackage{lmodern}
\usepackage[ngerman]{babel}
\usepackage[ngerman=ngerman-x-latest]{hyphsubst}

\usepackage{amsmath, amssymb, amsfonts, amsthm}
\usepackage{array}
\usepackage{stmaryrd}
\usepackage{marvosym}
\allowdisplaybreaks
\usepackage[output-decimal-marker={,}]{siunitx}
\usepackage[shortlabels]{enumitem}
\usepackage[section]{placeins}
\usepackage{float}
\usepackage{units}

\newtheorem*{behaupt}{Behauptung}
\newcommand{\gdw}{\Leftrightarrow}
\newcommand{\N}{\mathbb{N}}
\newcommand{\cov}{\operatorname{Cov}}
\newcommand{\e}{\operatorname{E}}
\newcommand{\var}{\operatorname{Var}}

\usepackage{fancyhdr}
\pagestyle{fancy}
\lhead{Stochastik 1 - Gruppe 5 - Blatt 8}
\rhead{Lennart Braun, Merlin Koglin, Kai Robin Sachse}
\cfoot{\thepage}


\title{Stochastik 1 für Informatiker}
\subtitle{Blatt 8 Hausaufgaben (Gruppe 5)}
\author{
    Lennart Braun (6523742), \\
    Merlin Koglin (6450362), \\
    Kai Robin Sachse (6450486)
}
\date{zum 16. Juni 2015}

\begin{document}
\maketitle

\begin{enumerate}[label=\bfseries\arabic*.]
    \item
        Sei $X$ eine auf $\{-10, -9, \ldots, -1, 0, 1, \ldots, 9, 10\}$
        gleichverteilte Zufallsvariable.
        \begin{behaupt}
            Es gelten $\e(X) = ?$ und $\var(X) = ?$.
        \end{behaupt}
        \begin{proof}
            TODO
        \end{proof}

    \item
        \begin{enumerate}[label=(\alph*)]
            \item
                Seien $X$, $Y$, $Z$ reellwertige Zufallsvariablen auf einem
                diskreten Wahrscheinlichkeitsraum $(\Omega, P)$ und
                $a, b \in \mathbb{R}$.
                \begin{behaupt}
                    Sofern alle Ausdrücke existieren gilt
                    \begin{equation*}
                        \cov(X, a \cdot Y + b \cdot Z)
                        = a \cdot \cov(X, Y) + b \cdot \cov(X, Z)
                    \end{equation*}
                \end{behaupt}
                \begin{proof}
                    TODO
                \end{proof}

            \item
                Sei $X$ eine reellwertige Zufallsvariable auf einem diskreten
                Wahrscheinlichkeitsraum $(\Omega, P)$, deren Erwartungswert und
                Varianz existieren.
                Die Varianz sei positiv.
                Definiere $Y := a + b \cdot X$ für $a, b \in \mathbb{R}$ und
                $b \neq 0$.
                \begin{behaupt}
                    Es gilt
                    \begin{equation*}
                        Corr(X, Y) =
                        \begin{cases}
                             1 & \text{, wenn } b > 0 \\
                            -1 & \text{, wenn } b < 0 \\
                        \end{cases}
                    \end{equation*}
                \end{behaupt}
                \begin{proof}
                    TODO
                \end{proof}

        \end{enumerate}

    \item
        Seien $X$, $Y$ reellwertige Zufallsvariablen auf einem diskreten
        Wahrscheinlichkeitsraum $(\Omega, P)$, deren Erwartungswerte
        existieren.
        Es gelten $\e(X+Y) = 0$, $\e(X-Y) = 2$ und $\e(X \cdot Y) = 0$.
        \begin{behaupt}
            Wir können (k)eine Aussage über die stochastische Unabhängigkeit
            von $X$ und $Y$ treffen: ?
        \end{behaupt}
        \begin{proof}
            TODO
        \end{proof}

\end{enumerate}


\end{document}
