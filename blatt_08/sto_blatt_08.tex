\documentclass[a4paper]{scrartcl}

% font/encoding packages
\usepackage[utf8]{inputenc}
\usepackage[T1]{fontenc}
\usepackage{lmodern}
\usepackage[ngerman]{babel}
\usepackage[ngerman=ngerman-x-latest]{hyphsubst}

\usepackage{amsmath, amssymb, amsfonts, amsthm}
\usepackage{array}
\usepackage{stmaryrd}
\usepackage{marvosym}
\allowdisplaybreaks
\usepackage[output-decimal-marker={,}]{siunitx}
\usepackage[shortlabels]{enumitem}
\usepackage[section]{placeins}
\usepackage{float}
\usepackage{units}

\newtheorem*{behaupt}{Behauptung}
\newcommand{\gdw}{\Leftrightarrow}
\newcommand{\N}{\mathbb{N}}
\newcommand{\cov}{\operatorname{Cov}}
\newcommand{\e}{\operatorname{E}}
\newcommand{\var}{\operatorname{Var}}
\newcommand{\corr}{\operatorname{Corr}}

\usepackage{fancyhdr}
\pagestyle{fancy}
\lhead{Stochastik 1 - Gruppe 5 - Blatt 8}
\rhead{Lennart Braun, Merlin Koglin, Kai Robin Sachse}
\cfoot{\thepage}


\title{Stochastik 1 für Informatiker}
\subtitle{Blatt 8 Hausaufgaben (Gruppe 5)}
\author{
    Lennart Braun (6523742), \\
    Merlin Koglin (6450362), \\
    Kai Robin Sachse (6450486)
}
\date{zum 16. Juni 2015}

\begin{document}
\maketitle

\begin{enumerate}[label=\bfseries\arabic*.]
    \item
        Sei $X$ eine auf
        $X(\Omega) = \{-10, -9, \ldots, -1, 0, 1, \ldots, 9, 10\}$
        gleichverteilte Zufallsvariable.
        \begin{behaupt}
            Es gelten $\e(X) = 0$ und
            $\var(X) = \frac{110}{3} \approx \num{36,6666}$.
        \end{behaupt}
        \begin{proof}
            Da $X$ gleichverteilt ist, gilt $P\{X = x\} = \frac{1}{21}$ für alle
            $x \in X(\Omega)$.

            Erwartungswert:
            \begin{equation*}
                \begin{split}
                    \e(X) &= \sum_{x \in X(\Omega)} x \cdot P\{X = x\} 
                          = \frac{1}{21} \cdot \sum_{i=-10}^{10} i 
                          = 0
                \end{split}
            \end{equation*}
            Varianz:
            \begin{equation*}
                \begin{split}
                    \var(X) &= \e(X^2) - (\e(X))^2 \\
                            &= \e(X^2) \\
                            &= \sum_{x \in X(\Omega)} x^2 \cdot P\{X = x\}  \\
                            &= \sum_{i=-10}^{10} i^2 \cdot P\{X = i\} \\
                            &= \frac{1}{21} \cdot \sum_{i=-10}^{10} i^2 \\
                            &= \frac{1}{21} \cdot 2 \cdot \sum_{i=1}^{10} i^2 \\
                            &= \frac{2}{21} \cdot \frac{1}{6} \cdot 10 \cdot
                               (10 + 1) \cdot (2 \cdot 10 + 1) \\
                            &= \frac{2}{21} \cdot \frac{1}{6} \cdot 10 \cdot 11
                               \cdot 21 \\
                            &= \frac{110}{3} = 36 + \frac{2}{3}
                               \approx \num{36,6666}
                \end{split}
            \end{equation*}
            Wir wenden den Transformationssatz für Erwartungswerte an, um
            $\e(X^2)$ zu berechnen.
        \end{proof}

    \item
        \begin{enumerate}[label=(\alph*)]
            \item
                Seien $X$, $Y$, $Z$ reellwertige Zufallsvariablen auf einem
                diskreten Wahrscheinlichkeitsraum $(\Omega, P)$ und
                $a, b \in \mathbb{R}$.
                \begin{behaupt}
                    Sofern alle Ausdrücke existieren gilt
                    \begin{equation*}
                        \cov(X, a \cdot Y + b \cdot Z)
                        = a \cdot \cov(X, Y) + b \cdot \cov(X, Z)
                    \end{equation*}
                \end{behaupt}
                \begin{proof}
                    \begin{equation*}
                        \begin{split}
                            \cov(X, aY + bZ)
                            &= \e(X \cdot (aY + bZ)) - E(X) \cdot E(aY + bZ) \\
                            &= \e(aXY + bXZ) - E(X) \cdot E(aY + bZ) \\
                            &= \e(aXY) + \e(bXZ) - E(X)
                               \cdot (E(aY) + \e(bZ)) \\
                            &= \e(aXY) + \e(bXZ) - E(X)
                               \cdot E(aY) - \e(X) \cdot \e(bZ) \\
                            &= a\e(XY) + b\e(XZ) - E(X)
                               \cdot aE(Y) - \e(X) \cdot b\e(Z) \\
                            &= a \cdot (\e(XY) - \e(X) \cdot \e(Y))
                               + b \cdot (\e(XZ) - E(X) \cdot \e(Z)) \\
                            &= a \cdot \cov(X, Y) + b \cdot \cov(X, Z)
                        \end{split}
                    \end{equation*}
                \end{proof}

            \item
                Sei $X$ eine reellwertige Zufallsvariable auf einem diskreten
                Wahrscheinlichkeitsraum $(\Omega, P)$, deren Erwartungswert und
                Varianz existieren.
                Die Varianz sei positiv.
                Definiere $Y := a + b \cdot X$ für $a, b \in \mathbb{R}$ und
                $b \neq 0$.
                \begin{behaupt}
                    Es gilt
                    \begin{equation*}
                        Corr(X, Y) =
                        \begin{cases}
                             1 & \text{, wenn } b > 0 \\
                            -1 & \text{, wenn } b < 0 \\
                        \end{cases}
                    \end{equation*}
                \end{behaupt}
                \begin{proof}
                    Sei $b > 0$.
                    \begin{equation*}
                        \begin{split}
                            \corr(X,Y)
                            &= \corr(X, a + b \cdot X) \\
                            &= \frac{\cov(X, a + b \cdot X)}
                            {\sqrt{\var(X) \cdot \var(a + b \cdot X)}} \\
                            &= \frac{b \cdot \cov(X, X)}
                            {\sqrt{\var(X) \cdot \var(X) \cdot b^2}} \\
                            &= \frac{b \cdot \var(X)}
                            {\var(X) \cdot b} = 1
                        \end{split}
                    \end{equation*}
                    Sei $b < 0$.
                    \begin{equation*}
                        \begin{split}
                            \corr(X,Y)
                            &= \corr(X, a + b \cdot X) \\
                            &= \frac{\cov(X, a + b \cdot X)}
                            {\sqrt{\var(X) \cdot \var(a + b \cdot X)}} \\
                            &= \frac{b \cdot \cov(X, X)}
                            {\sqrt{\var(X) \cdot \var(X) \cdot b^2}} \\
                            &= \frac{-1 \cdot |b| \cdot \var(X)}
                            {\var(X) \cdot \sqrt{b^2}} \\
                            &= \frac{-1 \cdot |b| \cdot \var(X)}
                            {\var(X) \cdot |b|} = -1
                        \end{split}
                    \end{equation*}
                \end{proof}

        \end{enumerate}

    \item
        Seien $X$, $Y$ reellwertige Zufallsvariablen auf einem diskreten
        Wahrscheinlichkeitsraum $(\Omega, P)$, deren Erwartungswerte
        existieren.
        Es gelten
        \[\e(X+Y) = 0,\; \e(X-Y) = 2 \text{ und } \e(X \cdot Y) = 0\]
        \begin{behaupt}
            Wir können eine Aussage über die stochastische Unabhängigkeit
            von $X$ und $Y$ treffen: $X$ und $Y$ sind nicht stochastisch
            unabhängig.
        \end{behaupt}
        \begin{proof}
            Zunächst verwenden wir die Linearität des Erwartungswertes:
            \begin{align*}
                \e(X + Y) = 0 &\gdw \e(X) + \e(Y) = 0 \\
                \e(X - Y) = 0 &\gdw \e(X) - \e(Y) = 2
            \end{align*}
            Addieren wir die Gleichungen auf der rechten Seite, können wir
            $\e(X) = 1$ und $\e(Y) = -1$ folgern.
            \begin{equation*}
                \begin{split}
                    \e(X) + \e(Y) + \e(X) - \e(Y) &= 0 + 2 \\
                    \gdw 2\e(X) &= 2 \\
                    \gdw \e(X) &= 1 \qquad \Rightarrow \e(Y) = -1
                \end{split}
            \end{equation*}
            Angenommen, $X$ und $Y$ wären stochastisch unabhängig.
            Dann wären sie auch unkorrelliert und es gelte $Cov(X, Y) = 0$.
            \begin{equation*}
                \begin{alignedat}{2}
                    && \cov(X, Y) &= 0 \\
                    \gdw\ && \e(X \cdot Y) - \e(X) \cdot \e(Y) &= 0 \\
                    \gdw\ && \e(X \cdot Y) &= \e(X) \cdot \e(Y)
                \end{alignedat}
            \end{equation*}
            Setzen wir in der letzten Zeile die gegebenen bzw. berechneten
            Werte ein, erhalten wir
            \begin{equation*}
                0 = 1 \cdot (-1)
                \gdw 0 = -1 \text{ .}
            \end{equation*}
            Dies ist offensichtlich ein Widerspruch, daher muss die Annahme, $X$
            und $Y$ seien stochastisch unabhängig, falsch sein.
            $X$ und $Y$ sind also nicht stochastisch unabhängig.
        \end{proof}

\end{enumerate}


\end{document}
