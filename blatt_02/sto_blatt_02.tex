\documentclass[a4paper]{scrartcl}

% font/encoding packages
\usepackage[utf8]{inputenc}
\usepackage[T1]{fontenc}
\usepackage{lmodern}
\usepackage[ngerman]{babel}
\usepackage[ngerman=ngerman-x-latest]{hyphsubst}

\usepackage{amsmath, amssymb, amsfonts, amsthm}
\usepackage{stmaryrd}
\allowdisplaybreaks
\usepackage[shortlabels]{enumitem}
\usepackage[section]{placeins}

\newtheorem*{behaupt}{Behauptung}
\newcommand{\gdw}{\Leftrightarrow}

\usepackage{fancyhdr}
\pagestyle{fancy}
\lhead{Stochastik 1 - Gruppe 5 - Blatt Nr. 2}
\rhead{Lennart Braun, Merlin Koglin, Kai Robin Sachse}
\cfoot{\thepage}


\title{Stochastik 1 für Informatiker}
\subtitle{Blatt 2 Hausaufgaben (Gruppe 5)}
\author{
    Lennart Braun (6523742), \\
    Merlin Koglin (TODO), \\
    Kai Robin Sachse (6450486)
}
\date{zum 21. April 2015}

\begin{document}
\maketitle

\begin{enumerate}[label=\bfseries\arabic*.]
    \item Es sei $(\Omega, P)$ ein endlicher Wahrscheinlichkeitsraum.
        \begin{enumerate}[label=(\roman*)]
            \item
                \begin{behaupt}
                    Für die leere Menge $\emptyset$ gilt $P(\emptyset) = 0$.
                \end{behaupt}
                \begin{proof}
                    
                \end{proof}

            \item
                \begin{behaupt}
                    Für alle paarweise disjunkten Teilmengen
                    $A_1, \ldots, A_k \subset \Omega, k \in \mathbb{N}$ gilt
                    \begin{equation*}
                        P \left( \bigcup_{i=1}^k A_i \right)
                        = \sum_{i=1}^k P(A_i)
                    \end{equation*}
                    
                \end{behaupt}
                \begin{proof}
                    
                \end{proof}

            \item
                \begin{behaupt}
                    Für alle Teilmengen $A \subset \Omega$ gilt
                    $P(A^C) = 1 - P(a)$
                \end{behaupt}
                \begin{proof}
                    
                \end{proof}

            \setcounter{enumii}{4}
            \item
                \begin{behaupt}
                    Für alle Teilmengen $A, B \subset \Omega$ gilt
                    $P(A \cup B) = P(A) + P(B) - P(A \cap B)$
                \end{behaupt}
                \begin{proof}
                    
                \end{proof}

        \end{enumerate}


    \item

    \item

\end{enumerate}

\end{document}
