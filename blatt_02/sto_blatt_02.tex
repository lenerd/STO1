\documentclass[a4paper]{scrartcl}

% font/encoding packages
\usepackage[utf8]{inputenc}
\usepackage[T1]{fontenc}
\usepackage{lmodern}
\usepackage[ngerman]{babel}
\usepackage[ngerman=ngerman-x-latest]{hyphsubst}

\usepackage{amsmath, amssymb, amsfonts, amsthm}
\usepackage{stmaryrd}
\allowdisplaybreaks
\usepackage[output-decimal-marker={,}]{siunitx}
\usepackage[shortlabels]{enumitem}
\usepackage[section]{placeins}

\newtheorem*{behaupt}{Behauptung}
\newcommand{\gdw}{\Leftrightarrow}

\usepackage{fancyhdr}
\pagestyle{fancy}
\lhead{Stochastik 1 - Gruppe 5 - Blatt Nr. 2}
\rhead{Lennart Braun, Merlin Koglin, Kai Robin Sachse}
\cfoot{\thepage}


\title{Stochastik 1 für Informatiker}
\subtitle{Blatt 2 Hausaufgaben (Gruppe 5)}
\author{
    Lennart Braun (6523742), \\
    Merlin Koglin (TODO), \\
    Kai Robin Sachse (6450486)
}
\date{zum 21. April 2015}

\begin{document}
\maketitle

\begin{enumerate}[label=\bfseries\arabic*.]
    \item Es sei $(\Omega, P)$ ein endlicher Wahrscheinlichkeitsraum.
        \begin{enumerate}[label=(\roman*)]
            \item
                \begin{behaupt}
                    Für die leere Menge $\emptyset$ gilt $P(\emptyset) = 0$.
                \end{behaupt}
                \begin{proof}
                    Angenommen es gelte $P(\emptyset) = k \neq 0$ mit
                    $k \in (0, 1]$.
                    Da es sich bei $(\Omega, P)$ um einen endlichen
                    Wahrscheinlichkeitsraum handelt, gilt für alle disjunkten
                    Teilmengen $A, B \subset \Omega$
                    $P(A \cup B) = P(A) + B(B)$.
                    Die leere Menge $\emptyset$ und der Grundraum $\Omega$ sind
                    solche Mengen: $\emptyset, \Omega \subset \Omega$.
                    \begin{equation}
                        \begin{split}
                            P(\Omega) &= P(\Omega \cup \emptyset) \\
                            \gdw P(\Omega) &= P(\Omega) + P(\emptyset) \\
                            \gdw P(\Omega) &= 1 + k > 1 \quad \lightning
                        \end{split}
                    \end{equation}
                    Obige Annahme, dass $P(\emptyset) \neq 0$ sei, führt zum
                    Widerspruch mit der Eigenschaft $P(\Omega) = 1$ von
                    endlichen Wahrscheinlichkeitsräumen.
                    Die Behauptung muss daher gelten.
                \end{proof}
                
                Würde ich eher direkt statt per Widerspruch zeigen. In etwa so: $P(\Omega)=1$ also gilt: $P(\Omega)=P(\Omega\cup\emptyset)=P(\Omega)+P(\emptyset)=1\Leftrightarrow P(\Omega)+P(\emptyset)-P(\Omega)=1-P(\Omega)\Leftrightarrow P(\emptyset)=1-1=0$

            \item
                \begin{behaupt}
                    Für alle paarweise disjunkten Teilmengen
                    $A_1, \ldots, A_k \subset \Omega, k \in \mathbb{N}$ gilt
                    \begin{equation*}
                        P \left( \bigcup_{i=1}^k A_i \right)
                        = \sum_{i=1}^k P(A_i)
                    \end{equation*}
                    
                \end{behaupt}
                \begin{proof}[Beweis durch vollständige Induktion] \hfill \\
                    \textbf{Induktionsanfang} \\
                    Sei $k = 1$.
                    \begin{equation}
                        \begin{split}
                            P(A_1) &= P(A_1) \\
                            \gdw P \left( \bigcup_{i=1}^1 A_i \right)
                            &= P(A_1) = \sum_{i=1}^1 P(A_i) \\
                            \gdw P \left( \bigcup_{i=1}^1 A_i \right)
                            &= \sum_{i=1}^1 P(A_i)
                        \end{split}
                    \end{equation}
                    
                    \textbf{Induktionsschritt} \\
                    Angenommen, es gelte die Behauptung für ein festes
                    $k \in  \mathbb{N}$.
                    \begin{equation}
                        \begin{split}
                            P \left( \bigcup_{i=1}^{k+1} A_i \right)
                            &= P \left( \left( \bigcup_{i=1}^k A_i \right)
                            \cup A_{k+1} \right) \\
                            \gdw P \left( \bigcup_{i=1}^{k+1} A_i \right)
                            &= P \left( \bigcup_{i=1}^k A_i \right) + P (A_{k+1})
                            \qquad \text{(da }\bigcup_{i=1}^{k} A_i \text{ und } A_{k+1} \text{ disjunkt)} \\
                            \gdw P \left( \bigcup_{i=1}^{k+1} A_i \right)
                            &\stackrel{I.A.}{=} \left( \sum_{i=1}^k P(A_i) \right)
                            + P (A_{k+1}) \\
                            \gdw P \left( \bigcup_{i=1}^{k+1} A_i \right)
                            &= \sum_{i=1}^{k+1} P(A_i)
                        \end{split}
                    \end{equation}
                    Dann gilt die Behauptung auch für $k+1$ und nach
                    vollständiger Induktion für alle $n \in \mathbb{N}$.
                    
                    
                \end{proof}

            \item
                \begin{behaupt}
                    Für alle Teilmengen $A \subset \Omega$ gilt
                    $P(A^C) = 1 - P(A)$
                \end{behaupt}
                \begin{proof}
                    Sei $A$ eine beliebige Teilmenge des Grundraumes.
                    Da $A \cup A^C = \Omega$ und $A \cap A^C = \emptyset$, gilt
                    \begin{equation}
                        \begin{split}
                            P(\Omega) &= 1 \\
                            \gdw P(A \cup A^C) &= 1 \\
                            \gdw P(A) + P(A^C) &= 1 \\
                            \gdw P(A^C) &= 1 - P(A) \text{ .}
                        \end{split}
                    \end{equation}
                \end{proof}

            \setcounter{enumii}{4}
            \item
                \begin{behaupt}
                    Für alle Teilmengen $A, B \subset \Omega$ gilt
                    $P(A \cup B) = P(A) + P(B) - P(A \cap B)$
                \end{behaupt}
                \begin{proof}
                    $A$ und $B$ lassen sich in jeweils zwei disjunkte Teilmengen
                    zerlegen:
                    $A = (A \setminus B) \cup (A \cap B)$ und
                    $B = (B \setminus A) \cup (B \cap A)$.
                    Es gilt $(A \setminus B) \cap (B \setminus A) = \emptyset$.
                    \begin{equation}
                        \begin{split}
                            P(A \cup B)
                            &= P \Big( ((A \setminus B) \cup (A \cap B))
                            \cup ((B \setminus A) \cup (B \cap A)) \Big) \\
                            \gdw P(A \cup B)
                            &= P \Big( (A \setminus B) \cup (A \cap B)
                            \cup (B \setminus A) \Big) \\
                            \gdw P(A \cup B)
                            &= P(A \setminus B) + P(A \cap B) + P(B \setminus A) \\
                            \gdw P(A \cup B) + P(A \cap B)
                            &= P(A \setminus B) + P(A \cap B)
                            + P(B \setminus A) + P(A \cap B) \\
                            \gdw P(A \cup B) + P(A \cap B)
                            &= P((A \setminus B) \cup (A \cap B))
                            + P((B \setminus A) \cup (B \cap A)) \\
                            \gdw P(A \cup B) + P(A \cap B) &= P(A) + P(B) \\
                            \gdw P(A \cup B) &= P(A) + P(B) -P(A \cap B)
                        \end{split}
                    \end{equation}
                    
                \end{proof}

        \end{enumerate}


    \item
        Wir nehmen an, Gummibärchen gleicher Farbe sind unterscheidbar.
        Es sei $R = \{ r_1, \ldots, r_4 \}$ die Menge der roten Gummibärchen
        in der Tüte.
        Analog sind $Ge = \{ ge_1, \ldots, ge_6 \}$ und
        $Gr = \{ gr_1, \ldots, gr_3 \}$ die Mengen der gelben bzw. grünen
        Gummibärchen.
        Es sei $B = R \cup Ge \cup Gr$.
        Wir definieren eine Ordnung auf $B$: Es gelten $ge < gr < r$ sowie
        $x_i < y_j \gdw x < y \lor (x = y \land i < j)$ für $x_i, y_j \in B$.
        Der Wahrscheinlichkeitsraum $(\Omega, P)$ sei wie folgt definiert.
        \begin{equation}
            \Omega = \left\{ (\omega_1, \omega_2, \omega_3) \in B^3 \ \vert \ 
            \omega_1 < \omega_2 < \omega_3 \right\}
            \qquad
            |\Omega| = \binom{13}{3} = 286
        \end{equation}
        Sei $(\omega_1, \omega_2, \omega_3) \in \Omega$, dann bedeutet
        $\omega_i = x_j$ für $i \in \{ 1, 2, 3 \}$, es wurde das Gummibärchen
        $x_j$ aus $B$ gezogen.
        $P$ sei das Laplace-Maß auf $\Omega$, d.\,h. es gelte
        $P(A) = \frac{|A|}{|\Omega|}$ für alle $A \subset \Omega$.

        Das Ereignis, dass drei Gummibärchen gleicher Farbe gezogen werden, wird
        durch $A = A_R \cup A_{Ge} \cup A_{Gr}$ beschrieben.
        $A_R$ sei das Ereignis, dass drei rote Gummibärchen gezogen werden.
        \begin{equation}
            A_R = \left\{ (\omega_1, \omega_2, \omega_3) \in R^3 \ \vert \ 
            \omega_1 < \omega_2 < \omega_3 \right\}
            \qquad
            |A_R| = \binom{4}{3} = 4
        \end{equation}
        $A_{Ge}$ und $A_{Gr}$ seien analog definiert.
        \begin{gather}
            A_{Ge} = \left\{ (\omega_1, \omega_2, \omega_3) \in Ge^3 \ \vert \ 
            \omega_1 < \omega_2 < \omega_3 \right\}
            \qquad
            |A_{Ge}| = \binom{6}{3} = 20 \\
            A_{Gr} = \left\{ (\omega_1, \omega_2, \omega_3) \in Gr^3 \ \vert \ 
            \omega_1 < \omega_2 < \omega_3 \right\}
            \qquad
            |A_{Gr}| = \binom{3}{3} = 1
        \end{gather}
        Da es sich um drei disjunkte Mengen handelt, gilt
        $|A| = |A_R| + |A_{Ge}| + |A_{Gr}| = 25$.
        Die Wahrscheinlichkeit, drei gleichfarbige Gummibärchen zu ziehen,
        beträgt daher $P(A) = \frac{25}{286} \approx \num{0,0874}$.
    
    \item
        \begin{behaupt}
            Es gibt 1001 Möglichkeiten die 10 Stimmen auf die 5 Parteien aufzuteilen.
        \end{behaupt}
        \begin{proof}
            Wir haben $k = 10$ Stimmen die wir auf $n = 5$ Parteien aufteilen
            wollen.
            Wir simulieren dies durch ein Fächermodell.
            Da eine Partei mehr als eine Stimme bekommen darf, darf jedes Fach
            mehrfach belegt werden.
            Ferner unterscheiden sich die Stimmen die abgegeben werden nicht,
            weshalb wir es mit ununterscheidbaren Murmeln zu tun haben.
            Die mögliche Anzahl der Stimmverteilungen bestimmt sich also durch
            \begin{equation}
                \binom{n+k-1}{k} = \binom{5+10-1}{10} = \binom{14}{10} = 1001
                \text{ .}
            \end{equation}
        \end{proof}

    

\end{enumerate}

\end{document}
