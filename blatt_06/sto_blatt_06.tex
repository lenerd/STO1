\documentclass[a4paper]{scrartcl}

% font/encoding packages
\usepackage[utf8]{inputenc}
\usepackage[T1]{fontenc}
\usepackage{lmodern}
\usepackage[ngerman]{babel}
\usepackage[ngerman=ngerman-x-latest]{hyphsubst}

\usepackage{amsmath, amssymb, amsfonts, amsthm}
\usepackage{stmaryrd}
\usepackage{marvosym}
\allowdisplaybreaks
\usepackage[output-decimal-marker={,}]{siunitx}
\usepackage[shortlabels]{enumitem}
\usepackage[section]{placeins}
\usepackage{float}
\usepackage{units}

\newtheorem*{behaupt}{Behauptung}
\newcommand{\gdw}{\Leftrightarrow}
\newcommand{\N}{\mathbb{N}}

\usepackage{fancyhdr}
\pagestyle{fancy}
\lhead{Stochastik 1 - Gruppe 5 - Blatt 6}
\rhead{Lennart Braun, Merlin Koglin, Kai Robin Sachse}
\cfoot{\thepage}


\title{Stochastik 1 für Informatiker}
\subtitle{Blatt 6 Hausaufgaben (Gruppe 5)}
\author{
    Lennart Braun (6523742), \\
    Merlin Koglin (6450362), \\
    Kai Robin Sachse (6450486)
}
\date{zum 19. Mai 2015}

\begin{document}
\maketitle

\begin{enumerate}[label=\bfseries\arabic*.]
    \item
        Seien $X$ eine $\mathcal{B}_{(n_1, p)}$-verteilte Zufallsvariable und
        $Y$ eine $\mathcal{B}_{(n_2, p)}$-verteilte Zufallsvariable mit
        $n_1, n_2 \in \mathbb{N}$ und $p \in [0,1]$.
        Ferner seien $X, Y$ stochastisch unabhängig.
        \begin{behaupt}
        	$X+Y$ ist $\mathcal{B}_{(n_1 + n_2, p)}$ verteilt.
        \end{behaupt}
        \begin{proof}
            Hier ein heuristischer Beweis: $X$ beschreibt die $n_1$-malige
            unabhängige Ausführung eines Zufallsexperiment mit der
            Erfolgswahrscheinlichkeit $p$.
            Analog beschreibt $Y$ die $n_2$-malige Ausführung.
            $X+Y$ beschreibt nun gerade die $n_1 + n_2$-malige Ausführung dieses
            Experiments.
            Damit ist die Verteilung gegeben durch
            $\mathcal{B}_{(n_1 + n_2, p)}$

            Eine weitere Möglichkeit ist der Beweis über Satz 3.9.
            \begin{equation*}
                \begin{split}
                    P^{X+Y}\{k\}
                    &= (P^X \ast P^Y) \{k\} \\
                    &= (\mathcal{B}_{(n_1, p)} \ast \mathcal{B}_{(n_2, p)})
                       \{k\} \\
                    &= (f_X \ast f_Y)(k) \\
                    &= \sum_{i=0}^k f_X(i) \cdot f_Y(k-i) \\
                    &= \sum_{i=0}^k \binom{n_1}{i} \cdot p^i \cdot (1-p)^{n_1-i}
                      \cdot \binom{n_2}{k-i} \cdot p^{k-i} \cdot (1-p)^{n_2-k+i}
                      \\
                    &= p^k \cdot (1-p)^{n_1 + n_2 - k} \cdot
                         \sum_{i=1}^k \binom{n_1}{i} \cdot \binom{n_2}{k-i} \\
                    &= \binom{n_1 + n_2}{k} \cdot p^k \cdot
                       (1-p)^{n_1 + n_2 - k} \\
                    &= \mathcal{B}_{(n_1 + n_2, p)}(\{k\})
                    \qedhere
                \end{split}
            \end{equation*}
        \end{proof}
        
    \item
        Bei einer Bestellung von $200$ Bauteilen sind \SI{5}{\percent} defekt.
        Bei einer Stichprobe werden $4$ Bauteile entnommen und geprüft.
        Befindet sich unter diesen mindestens ein defektes, so wird die
        Bestellung reklamiert.

    	\begin{behaupt}
            Die Zufallsvariable $X$ beschreibe die Anzahl defekter Bauteile
            unter den entnommenen.
            Die Verteilung von $X$ ist folgende hypergeometrische Verteilung.
            \begin{equation*}
                X \sim \mathcal{H}_{(200, 10, 4)}
            \end{equation*}
            Die Wahrscheinlichkeit für eine Reklamation (Ereignis $A$) liegt bei
            ca. $\num{0,1868}$.
    	\end{behaupt}
        \begin{proof}
            Wir betrachten die Grundmenge von $N = 200$ Bauteilen.
            Im Mittel sind $M = \num{0.05} \cdot N = 10$ Teile defekt, dies sind
            unsere markierten Bauteile und es werden $n = 4$ ohne Zurücklegen
            entnommen.
            Dieses Schema entspricht der hypergeometrischen Verteilung mit
            den gegebenen Parametern.
            \begin{equation*}
                X \sim (\mathcal{H}_{(N, M, n)} = \mathcal{H}_{(200, 10, 4)})
            \end{equation*}
            \begin{equation*}
                \begin{split}
                    P(A) &= P\{X \geq 1\} \\
                    &= 1 - P\{X = 0\} \\
                    &= 1 - \mathcal{H}_{(200, 10, 4)}\{0\} \\
                    &= 1 - \frac{\binom{10}{0} \cdot \binom{200 - 10}{4 - 0}}
                                {\binom{200}{4}} \\
                    &= 1 - \frac{\binom{190}{4}}{\binom{200}{4}} \\
                    &= 1 - \frac{190 \cdot \ldots \cdot 187}
                                {200 \cdot \ldots \cdot 197} \\
                    &\approx \num{0.1868}
                \end{split}
            \end{equation*}
            Mit einer Wahrscheinlichkeit von ca. \num{0,1868} wird die
            Bestellung reklamiert.
        \end{proof}
    	
    \item
        Eine zufällig ausgewählte Person hat mit einer Wahrscheinlichkeit von
        ca. \SI{1}{\percent} die Blutgruppe AB mit negativem Rhesusfaktor (AB-).
        \begin{enumerate}
            \item
                \begin{behaupt}
                    Die Wahrscheinlichkeit in einer Gruppe von 50 zufällig
                    ausgewählten Menschen mindestens eine Person mit dieser
                    Blutgruppe zu haben (Ereignis $A$) beträgt exakt
                    $1 - 0,99^{50} \approx 0,3950$.
                \end{behaupt}
                \begin{proof}
                    Sei $X \sim \mathcal{B}_{(50,\ \num{0,01})}$ die
                    Zufallsvariable, die die Anzahl der Menschen mit AB-
                    beschreibt.
                    Dann gilt
                    \begin{equation*}
                        \begin{split}
                            P(A) &= P\{X \geq 1\} \\
                            &= 1 - P\{X = 0\} \\
                            &= 1 - \mathcal{B}_{(50,\ \num{0,01})} (\{0\}) \\
                            &= 1 - \binom{50}{0} \cdot
                               \num{0,01}^0 \cdot (1 - \num{0,01})^{50 - 0} \\
                            &= 1 - \num{0,99}^{50} \\
                            &\approx \num{0.3950}
                        \end{split}
                    \end{equation*}
                    Da $n \geq 50$ und $p \leq \num{0,05}$, kann $X$ durch die
                    Poisson-Verteilung mit $\lambda = n \cdot p = \num{0,5}$
                    approximiert werden.
                    \begin{equation*}
                        \begin{split}
                            P(A) &= P\{X \geq 1\} \\
                            &= 1 - P\{X = 0\} \\
                            &\approx 1 - \mathcal{P}_{\num{0,5}} (\{0\}) \\
                            &= 1 - e^{-\num{0,5}} \cdot \frac{\num{0,5}^0}{0!} \\
                            &= 1 - e^{-\num{0,5}} \\
                            &\approx \num{0.3935}
                        \end{split}
                    \end{equation*}
                \end{proof}

            \item
                \begin{behaupt}
                    Ab einer Menge von $69$ zufällig gewählten Personen
                    befindet sich mit einer Wahrscheinlichkeit von mindestens
                    \SI{50}{\percent} mindestens eine Person mit AB- in dieser
                    Menge (Ereignis $A$).
                \end{behaupt}
                \begin{proof}
                    Sei $n$ die Anzahl der Personen, ab der die Wahrscheinlichkeit
                    über \SI{50}{\percent} liegt.
                    $n$ muss die kleinste Ganzzahl sein, welche die folgende
                    Ungleichung erfüllt.
                    \begin{equation*}
                        \begin{aligned}
                            && P(A) &\geq \num{0,5} \\
                            \gdw && P\{X \geq 1\} &\geq \num{0,5} \\
                            \gdw && 1 - P\{X = 0\} &\geq \num{0,5} \\
                            \gdw && P\{X = 0\} &\leq \num{0,5} \\
                            \gdw && \mathcal{B}_{(n,\ \num{0.01})}\{0\}
                                    &\leq \num{0,5} \\
                            \gdw && \binom{n}{0} \cdot \num{0,01}^0 \cdot
                                    (1 - \num{0,01})^{n - 0} &\leq \num{0,5} \\
                            \gdw && \num{0,99}^n &\leq \num{0,5} \\
                            \gdw && n \cdot \ln \num{0,99}
                                    &\leq \ln \num{0,5} \\
                            \gdw && n &\geq \frac{\ln \num{0,5}}{\ln \num{0,99}}
                                    \\
                            \gdw && n &\gtrapprox \num{68.9676}
                        \end{aligned}
                    \end{equation*}
                    Es ist also $n = \lceil \num{68,9676} \rceil = 69$.

                    Über die Approximierung durch die Poisson-Verteilung
                    $\mathcal{P}_{\num{0,5} \cdot n}$ kommt man zu einem
                    ähnlichen Ergebnis:
                    \begin{equation*}
                        \begin{aligned}
                            && P(A) \approx
                                    1 - \mathcal{P}_{\num{0.01} \cdot n}\{0\}
                                    &\geq \num{0,5} \\
                            \gdw && \mathcal{P}_{\num{0.01} \cdot n}\{0\}
                                    &\leq \num{0,5} \\
                            \gdw && e^{\num{-0.01} \cdot n} \cdot
                                    \frac{(\num{0,01} \cdot n)^0}{0!}
                                    &\leq \num{0,5} \\
                            \gdw && e^{\num{-0.01} \cdot n} &\leq \num{0,5} \\
                            \gdw && \num{-0.01} \cdot n &\leq \ln \num{0,5} \\
                            \gdw && n &\geq \frac{\ln \num{0,5}}{\num{-0,01}}
                                    \\
                            \gdw && n &\gtrapprox \num{69.3147}
                        \end{aligned}
                    \end{equation*}
                \end{proof}

        \end{enumerate}

\end{enumerate}


\end{document}
