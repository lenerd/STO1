\documentclass[a4paper]{scrartcl}

% font/encoding packages
\usepackage[utf8]{inputenc}
\usepackage[T1]{fontenc}
\usepackage{lmodern}
\usepackage[ngerman]{babel}
\usepackage[ngerman=ngerman-x-latest]{hyphsubst}

\usepackage{amsmath, amssymb, amsfonts, amsthm}
\usepackage{stmaryrd}
\usepackage{marvosym}
\allowdisplaybreaks
\usepackage[output-decimal-marker={,}]{siunitx}
\usepackage[shortlabels]{enumitem}
\usepackage[section]{placeins}
\usepackage{float}
\usepackage{units}

\newtheorem*{behaupt}{Behauptung}
\newcommand{\gdw}{\Leftrightarrow}
\newcommand{\N}{\mathbb{N}}

\usepackage{fancyhdr}
\pagestyle{fancy}
\lhead{Stochastik 1 - Gruppe 5 - Blatt 6}
\rhead{Lennart Braun, Merlin Koglin, Kai Robin Sachse}
\cfoot{\thepage}


\title{Stochastik 1 für Informatiker}
\subtitle{Blatt 6 Hausaufgaben (Gruppe 5)}
\author{
    Lennart Braun (6523742), \\
    Merlin Koglin (6450362), \\
    Kai Robin Sachse (6450486)
}
\date{zum 19. Mai 2015}

\begin{document}
\maketitle

\begin{enumerate}[label=\bfseries\arabic*.]
    \item
        Sei $X$ eine $\mathcal{B}_{\{n_1,p\}}$-vtlt Zv und $Y$ eine
        $\mathcal{B}_{\{n_2,p\}}$-vtlt Zv. mit $n_1,n_2\in\N$ und
        $p\in [0,1]$. Ferner seien $X,Y$ stochastisch unabhängig.
        \begin{behaupt}
        	$X+Y$ ist dann $\mathcal{B}_{\{n_1+n_2,p\}}$ verteilt.
        \end{behaupt}
        \begin{proof}
            Hier ein heuristischer Beweis:$X$ beschreibt die $n_1$ malige unabh.
            Ausführung eines Zufallsexperiment mit Erfolgswahrscheinlichkeit $p$.
            Analog $Y$ für $n_2$ malige Ausführung. $X+Y$ beschreibt nun gerade
            die $n_1+n_2$ malige Ausführung dieses Experiments. Damit ist die
            Verteilung gegeben durch $\mathcal{B}_{\{n_1+n_2,p\}}$
        \end{proof}
        
    \item
    	\begin{behaupt}
    		Bei einer Bestellung von $200$ Bauteilen sind $5$ Prozent defekt.
    		Bei einer Stichprobe werden $4$ Bauteile entnommen und geprüft.
    		Befindet sich unter diesen mindesten ein Defektes, so wird die
    		Bestellung Reklamiert (Ereignis A). Die Verteilung der entnommenen
    		defekten Bauteile ist gegeben durch:
    		\[f(x)=\frac{ \binom{10}{k} \binom{190}{4-k} }{\binom{200}{4}}\]
    		Die W. für eine Reklamation liegt bei $0,1868$
    	\end{behaupt}
    	\begin{proof}
    		Wir betrachten die $200$ Bauteile als unsere Grundmenge, die defekten
    		$200*0,05=10$ Bauteile als die markierten Bauteile und die $4$
    		entnommenen als unsere zweite Ziehung. Somit sind die entnommenen
    		defekten Bauteile gerade hypergeometrisch verteilt. Also gilt
    		\[f_X(k)=\frac{\binom{10}{k}\binom{190}{4-k}}{\binom{200}{4}}\]
    		Wobei die Zv. $X$ die Verteilung der entnommenen defekten Bauteile
    		bezeichne.\\
    		Die W. einer Reklamation ist nun
    		\[P(A)=P\{X\ge 1\}=1-P\{X=0\}=f_X(0)=1-\frac{\binom{10}{0}
    		\binom{190}{4-0}}{\binom{200}{4}}\approx 0,1868\]
    	\end{proof}
    	
    \item
    	\begin{behaupt}
    		Eine zufällig ausgewählte Person hat mit W. 0,01 die Blutgruppe AB-.
    		Die W. in einer Gruppe von 50 Menschen min eine Person mit dieser
    		Blutgruppe zu haben (Ereignis A) beträgt $1-0,99^{50}\approx 0,3950$
    	\end{behaupt}
    	\begin{proof}
    		Sei $X$ die Zv. die die Anzahl der AB- Menschen beschreibt. Dann
    		gilt $P(A)=P\{X\ge 0\}=1-P\{X=0\}=1-0,99^{50}\approx 0,3950$\\
    		Ab einer Menge von 69 Personen befindet sich mit einer W. von min
    		0,5 eine Person mit AB- in der Menge\\
    		Wir betrachten obige Zähldichte in Abhängigkeit von $k$:
    		$f_X(k)=1-0,99^k=0,5$ die Tabelle (TODO?) zeigt, dass dies für
    		$\frac{\log(2)}{\log(\frac{100}{99}}\approx 68,9$ erfüllt ist.
    		Um min. 0,5 zuerreichen braucht es also eine Menge von 69 Personen
    	\end{proof}
        
\end{enumerate}


\end{document}
