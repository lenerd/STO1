\documentclass[a4paper]{scrartcl}

% font/encoding packages
\usepackage[utf8]{inputenc}
\usepackage[T1]{fontenc}
\usepackage{lmodern}
\usepackage[ngerman]{babel}
\usepackage[ngerman=ngerman-x-latest]{hyphsubst}

\usepackage{amsmath, amssymb, amsfonts, amsthm}
\usepackage{stmaryrd}
\allowdisplaybreaks
\usepackage[output-decimal-marker={,}]{siunitx}
\usepackage[shortlabels]{enumitem}
\usepackage[section]{placeins}

\newtheorem*{behaupt}{Behauptung}
\newcommand{\gdw}{\Leftrightarrow}

\usepackage{fancyhdr}
\pagestyle{fancy}
\lhead{Stochastik 1 - Gruppe 5 - Blatt 3}
\rhead{Lennart Braun, Merlin Koglin, Kai Robin Sachse}
\cfoot{\thepage}


\title{Stochastik 1 für Informatiker}
\subtitle{Blatt 3 Hausaufgaben (Gruppe 5)}
\author{
    Lennart Braun (6523742), \\
    Merlin Koglin (6450362), \\
    Kai Robin Sachse (6450486)
}
\date{zum 28. April 2015}

\begin{document}
\maketitle

\begin{enumerate}[label=\bfseries\arabic*.]
    \item
        \begin{behaupt}
            Sei $(\Omega, P)$ ein diskreter Wahrscheinlichkeitsraum.
            Sei $B \subset \Omega$ mit $P(B) > 0$ und $P(B^C) > 0$.
            Dann gilt für alle $A \subset \Omega$
            \begin{equation*}
                P(A\ |\ B) = 1 - P(A\ |\ B^C)
            \end{equation*}
        \end{behaupt}
        \begin{proof}
            
        \end{proof}

    \item
        Ein endlicher Wahrscheinlichkeitsraum $(\Omega, P)$ sei wie folgt
        definiert.
        \begin{equation*}
            \Omega = \left\{ (\omega_1, \omega_2, \omega_3)
                \in \{1, \ldots, 6\}^3 \right\}
        \end{equation*}
        Für $\omega = (\omega_1, \omega_2, \omega_3) \in \Omega$,
        $i \in \{1, 2, 3\}$ und $j \in \{1, \ldots, 6\}$ bedeutet
        $\omega_i = j$, dass im $i$-ten Wurf die Augenzahl gleich $j$ ist.
        $P$ sei das Laplace-Maß auf $\Omega$, d.\,h. für alle $A \subset \Omega$
        gilt $P(A) = \frac{|A|}{|\Omega|}$.

        $A_k$ für $k \in \{1, \ldots, 6\}$ sei definiert als das Ereignis, dass
        die Auganzahl im ersten Wurf gleich $k$ ist.
        \begin{equation*}
            A_k = \left\{ (\omega_1, \omega_2, \omega_3) \in \Omega \ \vert \ 
            \omega_1 = k \right\}
        \end{equation*}
        $B$ sei das Ereignis, dass die Summe aller drei Würfe $9$ beträgt.
        \begin{equation*}
            \begin{split}
                B &= \left\{ (\omega_1, \omega_2, \omega_3) \in \Omega \ \vert\ 
                    \sum_{i=1}^3 \omega_i = 9 \right\} \\
                    &= \Big\{ 
                            (1, 2, 6), (1, 3, 5), (1, 4, 4), (1, 5, 3),
                            (1, 6, 2), \\
                    &\qquad (2, 1, 6), (2, 2, 5), (2, 3, 4), (2, 4, 3),
                            (2, 5, 2), (2, 6, 1), \\
                    &\qquad (3, 1, 5), (3, 2, 4), (3, 3, 3), (3, 4, 2),
                            (3, 5, 1), \\
                    &\qquad (4, 1, 4), (4, 2, 3), (4, 3, 2), (4, 4, 1), \\
                    &\qquad (5, 1, 3), (5, 2, 2), (5, 3, 1), \\
                    &\qquad (6, 1, 2), (6, 2, 1)
                    \Big\}
            \end{split}
        \end{equation*}
        $A_k \cap B$ enthält genau die Tupel, welche in der $k$-ten Zeile von
        $B$ aufgeführt sind.
        \begin{align*}
            A_1 \cap B &= \left\{ (1, 2, 6), (1, 3, 5), (1, 4, 4), (1, 5, 3),
                                  (1, 6, 2) \right\} \\
            A_2 \cap B &= \left\{ (2, 1, 6), (2, 2, 5), (2, 3, 4), (2, 4, 3),
                                  (2, 5, 2), (2, 6, 1) \right\} \\
            A_3 \cap B &= \left\{ (3, 1, 5), (3, 2, 4), (3, 3, 3), (3, 4, 2),
                                  (3, 5, 1) \right\} \\
            A_4 \cap B &= \left\{ (4, 1, 4), (4, 2, 3), (4, 3, 2), (4, 4, 1)
                          \right\} \\
            A_5 \cap B &= \left\{ (5, 1, 3), (5, 2, 2), (5, 3, 1) \right\} \\
            A_6 \cap B &= \left\{ (6, 1, 2), (6, 2, 1) \right\}
        \end{align*}
        Die Wahrscheinlichkeit, dass $A_k$ eintritt, gegeben, dass $B$ gilt
        berechnet sich wie folgt.
        \begin{equation*}
            \begin{split}
                P(A_k\ |\ B) &= \frac{P(A_k \cap B)}{P(B)} \\
                &= \frac{|A_k \cap B| \cdot |\Omega|}{|\Omega| \cdot |B|} \\
                &= \frac{|A_k \cap B|}{|B|}
            \end{split}
        \end{equation*}
        \begin{align*}
            P(A_1\ |\ B) &= \frac{5}{25} = \num{0,20} \\
            P(A_2\ |\ B) &= \frac{6}{25} = \num{0,24} \\
            P(A_3\ |\ B) &= \frac{5}{25} = \num{0,20} \\
            P(A_4\ |\ B) &= \frac{4}{25} = \num{0,16} \\
            P(A_5\ |\ B) &= \frac{3}{25} = \num{0,12} \\
            P(A_6\ |\ B) &= \frac{2}{25} = \num{0,08} \\
        \end{align*}

    \item
        \begin{enumerate}[label=(\alph*)]
            \item
                Es Sei $\Omega = \mathbb{N}_0$ und $\lambda > 0$
                \begin{behaupt}
                    \begin{equation*}
                        f\colon \Omega \to \mathbb{R}, \qquad
                        f(\omega)
                        = e^{-\lambda} \cdot \frac{\lambda^\omega}{\omega!}
                    \end{equation*}
                    ist eine Zähldichte auf $\Omega$.
                \end{behaupt}
                \begin{proof}
                    $\Omega = \mathbb{N}_0$ ist eine abzählbare Menge.
                    Nach Definition 1.10 ist $f\colon \Omega \to [0,1]$ die
                    Zähldichte genau eines diskreten Wahrscheinlichkeitsmaßes
                    auf $\Omega$, wenn
                    $\sum\limits_{\omega \in \Omega} f(\omega) = 1$ erfüllt ist.
                    Zu zeigen ist, dass nicht nur
                    $f\colon \Omega \to \mathbb{R}$ sondern auch
                    \begin{itemize}
                        \item $f\colon \Omega \to [0,1]$ sowie
                        \item $\sum\limits_{\omega \in \Omega} f(\omega) = 1$
                    \end{itemize}
                    gelten.
                    \begin{enumerate}
                        \item
                            \begin{equation*}
                                \begin{split}
                                    \sum_{\omega \in \mathbb{N}_0} f(\omega)
                                    &= \sum_{\omega \in \mathbb{N}_0}
                                        e^{-\lambda} \cdot
                                        \frac{\lambda^\omega}{\omega!} \\
                                    \gdw
                                    \sum_{\omega \in \mathbb{N}_0} f(\omega)
                                    &= e^{-\lambda}
                                        \cdot \sum_{\omega \in \mathbb{N}_0}
                                        \frac{\lambda^\omega}{\omega!} \\
                                    \stackrel{(\star)}{\gdw}
                                    \sum_{\omega \in \mathbb{N}_0} f(\omega)
                                    &= e^{-\lambda} \cdot e^\lambda \\
                                    \gdw
                                    \sum_{\omega \in \mathbb{N}_0} f(\omega)
                                    &= 1
                                \end{split}
                            \end{equation*}
                            Bei $(\star)$ wurde die Reihendarstellung der
                            Exponentialfunktion ausgenutzt.

                        \item
                            $f$ bildet auf ein Produkt von nichtnegativen Zahlen
                            ab, welches daher auch nichtnegativ ist.
                            Es gilt also $f\colon \Omega \to [0, \infty)$.
                            Da die Summe über alle Urbilder von $f$ gleich $1$
                            ist und aus nichtnegativen Summanden besteht, kann
                            kein Summand einen Wert größer $1$ besitzen.
                            Daher gilt auch $f\colon \Omega \to [0,1]$.

                    \end{enumerate}
                    Es existiert also ein diskretes Wahrscheinlichkeitsmaß auf
                    $\Omega$, welches $f$ zur Zähldichte hat.
                \end{proof}

            \item
                Es sei $f$ unter (a) angegeben mit $\lambda = 2$.
                \begin{equation*}
                    f(\omega) = e^{-2} \cdot \frac{2^\omega}{\omega!}
                \end{equation*}
                Für das diskrete Wahrscheinlichkeitsmaß $P$ gilt nach Satz 1.10
                \begin{equation*}
                    P(A) = \sum_{\omega \in A} f(\omega)
                \end{equation*}
                \begin{behaupt}
                    Für die Wahrscheinlichkeiten $P(\{0, 1, 2\})$ und
                    $P(\{0\}\ |\ \{0, 1, 2\})$ gelten
                    \begin{align*}
                        P(\{0, 1, 2\}) &\approx \num{0,6767} \\
                        P(\{0\}\ |\ \{0, 1, 2\}) &\approx \num{0,2000}
                    \end{align*}
                \end{behaupt}
                \begin{proof}
                    \begin{equation*}
                        \begin{split}
                            P(\{0, 1, 2\}) &=
                            \sum_{\omega \in \{0, 1, 2\}} f(\omega) \\
                            &\approx \num{0,1353} + \num{0,2707} + \num{0.2707}
                            \\
                            &\approx \num{0,6767}
                        \end{split}
                    \end{equation*}
                    \begin{equation*}
                        \begin{split}
                            P(\{0\}\ |\ \{0, 1, 2\})
                            &= \frac{P(\{0\} \cap \{0, 1, 2\})}{P(\{0, 1, 2\})}
                            \\
                            &= \frac{P(\{0\})}{P(\{0, 1, 2\})} \\
                            &= \frac{\sum\limits_{\omega \in \{0\}} f(\omega)}
                                {\sum\limits_{\omega \in \{0, 1, 2\}} f(\omega)}
                                \\
                            &\approx \frac{\num{0.1353}}{\num{0.6767}}
                            \approx \num{0,2000}
                        \end{split}
                    \end{equation*}
                \end{proof}

        \end{enumerate}

\end{enumerate}

\end{document}
