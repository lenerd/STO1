\documentclass[a4paper]{scrartcl}

% font/encoding packages
\usepackage[utf8]{inputenc}
\usepackage[T1]{fontenc}
\usepackage{lmodern}
\usepackage[ngerman]{babel}
\usepackage[ngerman=ngerman-x-latest]{hyphsubst}

\usepackage{amsmath, amssymb, amsfonts, amsthm}
\usepackage{stmaryrd}
\allowdisplaybreaks
\usepackage[output-decimal-marker={,}]{siunitx}
\usepackage[shortlabels]{enumitem}
\usepackage[section]{placeins}

\newtheorem*{behaupt}{Behauptung}
\newcommand{\gdw}{\Leftrightarrow}

\usepackage{fancyhdr}
\pagestyle{fancy}
\lhead{Stochastik 1 - Gruppe 5 - Blatt 4}
\rhead{Lennart Braun, Merlin Koglin, Kai Robin Sachse}
\cfoot{\thepage}


\title{Stochastik 1 für Informatiker}
\subtitle{Blatt 4 Hausaufgaben (Gruppe 5)}
\author{
    Lennart Braun (6523742), \\
    Merlin Koglin (6450362), \\
    Kai Robin Sachse (6450486)
}
\date{zum 5. Mai 2015}

\begin{document}
\maketitle

\begin{enumerate}[label=\bfseries\arabic*.]
    \item
        \begin{behaupt}
            
        \end{behaupt}
        \begin{proof}
           
        \end{proof}

    \item	        
        	\begin{behaupt}
        	
        	\end{behaupt}
        	\begin{proof}
        	Wir berechnen zuerst die W. von $A,B$ und $C$:
			\begin{equation*}
			P(A)=\frac{4}{6}=\frac{2}{3};\; P(B)=P(C)=\frac{3}{6}=\frac{1}{2}
			\end{equation*}
			Wir bestimmen nun für jedes Paar den Schnitt und prüfen dann ob die Gleichung aus $2.6$ gilt:\\     
        	$A\cap B=\{4\}\Rightarrow P(A\cap B)=\frac{1}{6}\ne \frac{1}{3} = 
        	\frac{2}{3}\cdot \frac{1}{2} =P(A)\cdot P(B)$\\
        	$B\cap C=\{4,5,6\}\Rightarrow P(B\cap C)=\frac{3}{6}=\frac{1}{2} \ne \frac{1}{4} =
        	\frac{1}{2}\cdot \frac{1}{2} =P(B)\cdot P(C)$\\
        	$A\cap C=\{4\}\Rightarrow P(A\cap C)=\frac{1}{6}\ne \frac{1}{3} =
        	\frac{2}{3}\cdot \frac{1}{2} =P(A)\cdot P(C)$
        	\end{proof}
        	
        	
        	\begin{behaupt}
        	Es gilt $P(A\cap B\cap C)=P(A)\cdot P(B)\cdot P(C)$
        	\end{behaupt}
        	\begin{proof}
        	$A\cap B\cap C=\{4\}\Rightarrow P(A\cap B\cap C)=\frac{1}{6} = \frac{1}{6}=
        	\frac{2}{3}\cdot \frac{1}{2}\cdot \frac{1}{2}= P(A)\cdot P(B)\cdot P(C)$
        	\end{proof}
        	
        	\begin{behaupt}
        	Die drei Mengen sind nicht stochastisch abhängig.
        	\end{behaupt}
        	\begin{proof}
        	Um zu prüfen, ob eine Menge von Ereignisse st. unabh. ist, müssen wir jede
        	Teilmenge auf Unabhängigkeit prüfen. Da in a) aber schon $A$ und $B$ nicht
        	unabhängig sind, sind erst recht nicht $A$ und $B$ und $C$ unabhängig.
        	\end{proof}

    \item
                
                \begin{behaupt}
                	Die kostengünstigste Konstruktionsmöglichkeit,
					die die gewünschte Versorgungssicherheit gewährleistet,
					ist, wenn man die Strecken $a-c$ und $a-d$ mit der schlechtesten Qualität und
					die anderen beiden Streckenabschnitte mit der besten Qualität baut. Die Kosten
					belaufen sich dabei auf $400$ Euro.     
                \end{behaupt}
                \begin{proof}
                	Seien $A,B,C,D$ die Ereignisse, dass jeweils die Strecken
					$ac$,$cb$,$ad$ bzw. $db$ funktionieren. Dann beschreiben $A\cap B$ und 
					$C\cap D$ jeweils das Ereignis, dass die beiden Gesamtstrecken $ac-cb$ bzw.
					$ad-db$ funktionieren. Das Ereignis, dass mindestens eine Strecke
					funktioniert ist folglich $(A\cap B)\cup (C\cap D)$.
					Als Grundraum können wir $\Omega=\{0,1\}^4$ betrachten, wobei zB für $A$
					gilt $A=\{(w_1,w_2,w_3,w_4)\in\Omega|w_1=1\}$ sind. Da die Ereignisse
					$A,B,C,D$ stochastisch unabhängig sind, ergibt sich folgende Gleichung.
					\begin{align*}
					P((A\cap B)\cup (C\cap D))
					&=P(A\cap B)+P(C\cap D)-P(A\cap B\cap C\cap D)\\
					&=P(A)\cdot P(B)+P(C)\cdot P(D)-P(A)\cdot P(B)\cdot P(C)\cdot P(D)
					\end{align*}
					Durch geschicktes einsetzten erhält man, dass für:\\
					$P(A)=0.95\quad P(B)=0.99 \quad P(C)=0.95 \quad P(D)=0.99 \Rightarrow
					P((A\cap B)\cup (C\cap D))=0.99646$\\
					$P(A)=0.9\quad P(B)=0.99 \quad P(C)=0.9 \quad P(D)=0.99 \Rightarrow
					P((A\cap B)\cup (C\cap D))=0.996219$\\
					Wobei die untere Lösung deutlich Preiswerter ist mit $500$ Euro vs. $400$ Euro.\\
					($P(A)$ und $P(B)$ ($P(C)$ und $P(D)$) lassen sich natürlich vertauschen)
					Es gibt außerdem keine 
					billigere Lösung, da alle Kombinationen mit weniger als 2 von den 
					teuersten Strecken die gewünschte Wahrscheinlichkeit sowieso nicht erreichen.    
                \end{proof}

                

\end{enumerate}

\end{document}
