\documentclass[a4paper]{scrartcl}

% font/encoding packages
\usepackage[utf8]{inputenc}
\usepackage[T1]{fontenc}
\usepackage{lmodern}
\usepackage[ngerman]{babel}
\usepackage[ngerman=ngerman-x-latest]{hyphsubst}

\usepackage{amsmath, amssymb, amsfonts, amsthm}
\usepackage{stmaryrd}
\usepackage{marvosym}
\allowdisplaybreaks
\usepackage[output-decimal-marker={,}]{siunitx}
\usepackage[shortlabels]{enumitem}
\usepackage[section]{placeins}

\newtheorem*{behaupt}{Behauptung}
\newcommand{\gdw}{\Leftrightarrow}

\usepackage{fancyhdr}
\pagestyle{fancy}
\lhead{Stochastik 1 - Gruppe 5 - Blatt 4}
\rhead{Lennart Braun, Merlin Koglin, Kai Robin Sachse}
\cfoot{\thepage}


\title{Stochastik 1 für Informatiker}
\subtitle{Blatt 4 Hausaufgaben (Gruppe 5)}
\author{
    Lennart Braun (6523742), \\
    Merlin Koglin (6450362), \\
    Kai Robin Sachse (6450486)
}
\date{zum 5. Mai 2015}

\begin{document}
\maketitle

\begin{enumerate}[label=\bfseries\arabic*.]
    \item
        Sei $A$ das Ereignis, dass der gewählte Student die Antwort kennt und
        $B$ das Ereignis, dass der Student die richtige Antwort ankreuzt.
        Aus dem Text lassen sich folgende Informationen entnehmen:
        \begin{itemize}
            \item $P(A) = \num{0.7}$ und damit auch $P(A^C) = \num{0.3}$
            \item $P(B\ |\ A) = 1$
            \item $P(B\ |\ A^C) = \frac{1}{3}$
        \end{itemize}
        \begin{enumerate}[label=(\alph*)]
            \item
                \begin{behaupt}
                    Der Student kreuzt die richtige Antwort mit der
                    Wahrscheinlichkeit $P(B) = \num{0.8}$ an.
                \end{behaupt}
                \begin{proof}
                    Da $A$ und $A^C$ eine disjunkte Zerlegung von des
                    Grundraumes bilden, lässt sich $P(B)$ mit Hilfe des Satzes
                    der totalen Wahrscheinlichkeit berechnen.
                    \begin{equation*}
                        \begin{split}
                            P(B) &= P(B\ |\ A) \cdot P(A)
                                    + P(B\ |\ A^C) \cdot P(A^C) \\
                                 &= 1 \cdot \num{0.7}
                                    + \frac{1}{3} \cdot \num{0.3} \\
                                 &= \num{0.8}
                        \end{split}
                    \end{equation*}
                \end{proof}

            \item
                \begin{behaupt}
                    Wenn die angekreuzte Antwort korrekt ist, hat der Student
                    die richtige Antwort mit der Wahrscheinlichkeit
                    $P(A\ |\ B) = \frac{7}{8}$ gewusst.
                \end{behaupt}
                \begin{proof}
                    Da $A$ und $A^C$ eine disjunkte Zerlegung von des
                    Grundraumes bilden, lässt sich der Satz von Bayes anwenden,
                    um $P(A\ |\ B)$ zu berechnen.
                    \begin{equation*}
                        \begin{split}
                            P(A\ |\ B) &= \frac{P(B\ |\ A) \cdot P(A)}
                            {P(B\ |\ A) \cdot P(A) + P(B\ |\ A^C) \cdot P(A^C)}
                            \\
                            &= \frac{P(B\ |\ A) \cdot P(A)}{P(B)} \\
                            &= \frac{1 \cdot \num{0.7}}{\num{0.8}} \\
                            &= \frac{7}{8}
                        \end{split}
                    \end{equation*}
                \end{proof}

        \end{enumerate}

    \item	        
        \begin{enumerate}[label=(\alph*)]
            \item
                \begin{behaupt}
                    Die Ereignisse $A, B$, $B, C$ und $A, C$ sind jeweils nicht
                    stochastisch unabhängig.
                \end{behaupt}
                \begin{proof}
                Wir berechnen zuerst die Wahrscheinlichkeiten der drei
                Ereignisse.
                \begin{align*}
                    P(A) = \frac{4}{6} = \frac{2}{3}
                    \qquad
                    P(B) = P(C) = \frac{3}{6} = \frac{1}{2}
                \end{align*}
                Wir bestimmen nun für jedes Paar den Schnitt und prüfen dann ob
                die Gleichung aus $2.6$ gilt.
                \begin{gather*}
                    A \cap B = \{4\} \Rightarrow P(A \cap B) = \frac{1}{6} \neq
                    \frac{1}{3} = \frac{2}{3} \cdot \frac{1}{2}
                    = P(A) \cdot P(B)
                    \\
                    B \cap C =\{4,5,6\} \Rightarrow P(B \cap C) = \frac{3}{6} =
                    \frac{1}{2} \neq \frac{1}{4} = \frac{1}{2} \cdot \frac{1}{2}
                    = P(B) \cdot P(C)
                    \\
                    A \cap C = \{4\} \Rightarrow P(A \cap C) = \frac{1}{6} \neq
                    \frac{1}{3} = \frac{2}{3} \cdot \frac{1}{2}
                    = P(A) \cdot P(C)
                \end{gather*}
                Da jeweils Ungleichheit besteht, die die drei Ereignisse
                paarweise nicht stochastisch unabhängig.
                \end{proof}
                
            \item
                \begin{behaupt}
                    Es gilt $P(A \cap B \cap C) = P(A) \cdot P(B) \cdot P(C)$.
                \end{behaupt}
                \begin{proof}
                    \begin{equation*}
                        A \cap B \cap C = \{4\} \Rightarrow P(A\cap B\cap C)
                        = \frac{1}{6} = \frac{1}{6} = \frac{2}{3}
                        \cdot \frac{1}{2} \cdot \frac{1}{2}
                        = P(A) \cdot P(B) \cdot P(C)
                    \end{equation*}
                \end{proof}
                
            \item
                \begin{behaupt}
                    Die Mengen $A, B, C$ sind nicht stochastisch unabhängig.
                \end{behaupt}
                \begin{proof}
                    Um zu prüfen, ob eine Menge von Ereignissen stochastisch
                    unabhängig ist, müssen wir jede Teilmenge auf Unabhängigkeit
                    prüfen.
                    Da in (a) aber schon $A$ und $B$ nicht unabhängig sind, sind
                    erst recht nicht $A$ und $B$ und $C$ stochastisch
                    unabhängig.
                \end{proof}
        \end{enumerate}

    \item
                
                \begin{behaupt}
                	% Die kostengünstigste Konstruktionsmöglichkeit,
                	% die die gewünschte Versorgungssicherheit gewährleistet,
                	% ist, wenn man die Strecken $a-c$ und $a-d$ mit der schlechtesten Qualität und
                	% die anderen beiden Streckenabschnitte mit der besten Qualität baut. Die Kosten
                	% belaufen sich dabei auf $400$ Euro.     
                    Die preiswerteste Konstruktion, welche mit einer
                    Wahrscheinlichkeit von mindestens \num{0.995} die
                    Stromversorgung sichert verwendet auf einer Strecke die
                    teuersten Verbindungen hintereinander und auf der anderen
                    Strecke die billigsten Verbindungen.
                    Die Kosten betragen $400\,\text{\EUR}$.
                \end{behaupt}
                \begin{proof}
                    Seien $A, B, C, D$ die Ereignisse, dass jeweils die Strecken
                    $ac$, $cb$, $ad$ bzw. $db$ funktionieren.
                    Dann beschreiben $A\cap B$ und $C\cap D$ jeweils das
                    Ereignis, dass die beiden Gesamtstrecken $ac-cb$ bzw.
                    $ad-db$ funktionieren.
                    Das Ereignis, dass mindestens eine Strecke
                    funktioniert ist folglich $(A \cap B)\cup (C\cap D)$.
                    Als Grundraum können wir $\Omega = \{0, 1\}^4$ betrachten,
                    wobei z.\,B. für $A$ gilt
                    $A = \{(w_a, w_b, w_c, w_d) \in \Omega \ |\ w_a = 1\}$ sind.
                    Da die Ereignisse $A, B, C, D$ stochastisch unabhängig sind,
                    ergibt sich folgende Gleichung:
                    \begin{equation*}
                        \begin{split}
                            P((A \cap B) \cup (C \cap D))
                            &= P(A \cap B) + P(C \cap D)
                               - P(A \cap B \cap C \cap D) \\
                            &= P(A) \cdot P(B) + P(C) \cdot P(D)
                               - P(A) \cdot P(B) \cdot P(C) \cdot P(D)
                        \end{split}
                    \end{equation*}
                    Durch geschicktes einsetzten erhält man, dass für:\\
                    $P(A)=0.95\quad P(B)=0.99 \quad P(C)=0.95 \quad P(D)=0.99 \Rightarrow
                    P((A\cap B)\cup (C\cap D))=0.99646$\\
                    $P(A)=0.9\quad P(B)=0.99 \quad P(C)=0.9 \quad P(D)=0.99 \Rightarrow
                    P((A\cap B)\cup (C\cap D))=0.996219$\\
                    % Habe da andere Werte raus:
                    % In [20]: 0.9 * 0.99 + 0.9 * 0.99 - 0.9 * 0.99 * 0.9 * 0.99
                    % Out[20]: 0.988119
                    
                    % Merlin:
                    % Ich denke Kairo hat sich hier verschrieben und meinte: 
                    % P(A)=0.9 P(B)=0.9 P(C)=0.99 P(D)=0.99 bzw
                    % P(A)=0.99 P(B)=0.99 P(C)=0.9 P(D)=0.9
                    % Das entspricht seiner (und auch deiner) kostenoptimalen Konstruktion.
                   
                    Wobei die untere Lösung deutlich Preiswerter ist mit $500$ Euro vs. $400$ Euro.\\
                    ($P(A)$ und $P(B)$ ($P(C)$ und $P(D)$) lassen sich natürlich vertauschen)
                    Es gibt außerdem keine 
                    billigere Lösung, da alle Kombinationen mit weniger als 2 von den 
                    teuersten Strecken die gewünschte Wahrscheinlichkeit sowieso nicht erreichen.    
                \end{proof}

             

                \begin{proof}
                    Es sei $L = \{1, 2, 3\}$ die Menge der Leitungstypen.
                    Im Folgenden bezeichnen $A, B, C, D$ die Kanten
                    $(a, c), (c, b), (a, d), (d, b)$.

                    Die Menge der möglichen Konfigurationen sei $C = L^4$.%Hier solltest du statt C eine andere Benennung wählen, C hast du ja schon als Kante definiert:)
                    Für $(A, B, C, D) \in C$ bedeutet $A = 2$, dass die 
                    Leitung $A$ in der Ausführung 2 für 100 \EUR gebaut wird und
                    mit einer Wahrscheinlichkeit von \num{0,05} ausfällt.
                    Für unsere Rechnung sind in $C$ jedoch redundante Fälle
                    enthalten, welche durch eine andere Benennung in einander
                    übergehen:
                    Für die Gesamtwahrscheinlichkeit macht keinen Unterschied ob
                    die Konfiguration $(1, 1, 2, 2)$ oder die Konfiguration
                    $(2, 2, 1, 1)$ gebaut wird.
                    Die obere und die untere Strecke sind also austauschbar.
                    Analog können auch die beiden Teilstücke der Leitungen
                    vertauscht werden, sodass $(1, 2, 3, 3)$ und $(2, 1, 3, 3)$
                    äquivalent sind.
                    Wir können also ohne Einschränkung die Menge an
                    verschiedenen Konfigurationen $C'$ definieren:
                    \begin{equation*}
                        C' := \Big\{ 
                            (A, B, C, D) \in C \ |\ 
                            (p_A \leq p_B) \land (p_C \leq p_D) \land
                            (p_A \cdot p_B \leq p_C \cdot p_D)
                        \Big\}
                    \end{equation*}
                    Es werden also die Konfigurationen betrachtet, in denen
                    jeweils die erste der beiden Teilstrecken mindestens so
                    verlässlich ist wie die zweite.
                    Das sind jeweils sechs verschiedene Möglichkeiten
                    (Urnenmodell mit drei Kugeln: Ziehen von zwei Kugeln mit
                    Zurücklegen und ohne Berücksichtigung der Reihenfolge).
                    Da auch die obere und die untere Leitung vertauscht werden
                    können, fordern wir, dass die obere Leitung mindestens so
                    verlässlich ist wie die untere.
                    Damit gibt es $|C'| = 21$ Möglichkeiten.
                    (Urnenmodell mit sechs Kugeln: Ziehen von zwei Kugeln mit
                    Zurücklegen und ohne Berücksichtigung der Reihenfolge).

                    Die Ereignisse, dass Leitungen (nicht) ausfallen sind
                    stochastisch unabhängig von einander.
                    Im Kontext der Wahrscheinlichkeiten bezeichnen wir mit $A$
                    das Ereignis, dass die dazugehörige Strecke $(a, c)$ nicht
                    ausfällt (analog für $B, C, D$). 
                    Dann beschreiben $A \cap B$ und $C \cap D$ die Ereignisse,
                    dass die obere bzw. untere Strecke nicht unterbrochen ist.
                    Das Ereignis, dass mindestens eine der beiden Strechen
                    funktioniert ist folglich $(A \cap B) \cup (C \cap D)$.
                    Als Grundraum betrachten wir $\Omega = \{0, 1\}^4$.
                    Für $A$ gilt $A = \{(\omega_A, \omega_B, \omega_C, \omega_D)
                    \in \Omega \ |\ \omega_A = 1\}$ ($B, C, D$ analog).
                    Das Wahrscheinlichkeitsmaß $P$ ergibt sich aus der
                    verwendeten Konfiguration $c \in C'$.
                    Ist $c = (A, B, C, D)$ so ist $P(A) = 1 - p_A$ aus der
                    Aufgabenstellung.
                    \begin{equation*}
                        \begin{split}
                            P((A \cap B) \cup (C \cap D))
                            &= P(A \cap B) + P(C \cap D)
                               - P((A \cap B) \cap (C \cap D)) \\
                            &= P(A \cap B) + P(C \cap D)
                               - P(A \cap B) \cdot P(C \cap D) \\
                            &= P(A \cap B) + P(C \cap D)
                               \cdot (1 - P(A \cap B))
                       \end{split}
                    \end{equation*}
                    Aus der Wahl unserer Konfiguration folgt:
                    \begin{equation*}
                        \num{0.9}^2 \leq P(C \cap D) \leq P(A \cap B)
                    \end{equation*}
                    Wir können nun Abschätzungen für die
                    Gesamtwahrscheinlichkeit nach oben und unten vornehmen.
                    \begin{equation*}
                        P(A \cap B) + \num{0,9}^2 \cdot (1 - P(A \cap B))
                        \leq P((A \cap B) \cup (C \cap D))
                        \leq P(A \cap B) + P(A \cap B) \cdot (1 - P(A \cap B))
                    \end{equation*}
                    Die Werte sind in Tabelle \ref{tab:absch} aufgetragen.
                    Wir können dieser entnehmen, dass wir die gewünschte
                    Verlässlichkeit nur erreichen können, wenn
                    $(A, B) = (3, 2)$ oder $(A, B) = (3, 3)$ gilt.
                    \begin{table}
                        \centering
                        \begin{tabular}{l|l|l|l}
                            Konfiguration & $P(A \cap B)$ & $P_{min}$ & $P_{max}$ \\
                            \hline
                            $(1, 1, x, y)$ & \num{0.810} & \num{0.964} & \num{0.964} \\
                            $(2, 1, x, y)$ & \num{0.855} & \num{0.972} & \num{0.979} \\
                            $(3, 1, x, y)$ & \num{0.891} & \num{0.979} & \num{0.988} \\
                            $(2, 2, x, y)$ & \num{0.902} & \num{0.981} & \num{0.990} \\
                            $(3, 2, x, y)$ & \num{0.941} & \num{0.989} & \num{0.996} \\
                            $(3, 3, x, y)$ & \num{0.980} & \num{0.996} & \num{1.000} \\%*(siehe unten)
                        \end{tabular}
                        \caption{Abschätzungen der Gesamtwahrscheinlichkeit}
                        \label{tab:absch}
                    \end{table}
                    Damit haben wir die Möglichkeiten auf wenige Fälle
                    reduziert, welche in Tabelle \ref{tab:konf} dargestellt
                    sind.
                    Offensichtlich handelt es sich bei der Konfiguration
                    $(3, 3, 1, 1)$ um die günstigste derer, die die Anforderung
                    erfüllen.
                    Man sollte also die obere Leitung in der teuersten und die
                    untere in der billigsten Ausführung bauen
                    (oder andersherum).
                    \begin{table}
                        \centering
                        \begin{tabular}{l|l|l|l}
                            Konfiguration & $P((A \cap B) \cup (C \cap D))$ & Preis \\
                            \hline
                            $(3, 2, 1, 1)$ & \num{0.988695} & \num{350}\,\EUR \\
                            $(3, 2, 2, 1)$ & \num{0.991373} & \num{400}\,\EUR \\
                            $(3, 2, 3, 1)$ & \num{0.993515} & \num{450}\,\EUR \\
                            $(3, 2, 2, 2)$ & \num{0.994199} & \num{450}\,\EUR \\
                            $(3, 2, 3, 2)$ & \num{0.996460} & \num{500}\,\EUR \\
                            $(3, 3, 1, 1)$ & \num{0.996219} & \num{400}\,\EUR \\
                            $(3, 3, 2, 1)$ & \num{0.997115} & \num{450}\,\EUR \\
                            $(3, 3, 3, 1)$ & \num{0.997831} & \num{500}\,\EUR \\
                            $(3, 3, 2, 2)$ & \num{0.998060} & \num{500}\,\EUR \\
                            $(3, 3, 3, 2)$ & \num{0.998816} & \num{550}\,\EUR \\
                            $(3, 3, 3, 3)$ & \num{0.999604} & \num{600}\,\EUR \\
                        \end{tabular}
                        \caption{In Frage kommende Konfigurationen}
                        \label{tab:konf}
                    \end{table}
				% Diese Werte sind ja auf drei Nachkommastellen gerundet, dadurch ist
				% laut Tabelle(n) bei (3, 3, 3, 3) die Stromversorgung immer gesichert,
				% die Wahrscheinlichkeit dafür beträgt aber nur 0.99960399 ist also nicht 1.
                % Lennart: Da hatte ich beim Generieren der Tabelle nicht aufgepasst.
                \end{proof}

                

\end{enumerate}

\end{document}
