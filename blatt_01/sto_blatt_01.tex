\documentclass[a4paper]{scrartcl}

% font/encoding packages
\usepackage[utf8]{inputenc}
\usepackage[T1]{fontenc}
\usepackage{lmodern}
\usepackage[ngerman]{babel}

\usepackage{amsmath, amssymb, amsfonts, amsthm}
\usepackage{stmaryrd}
\allowdisplaybreaks
\usepackage[output-decimal-marker={,}]{siunitx}
\usepackage[shortlabels]{enumitem}
\usepackage[section]{placeins}

\newtheorem*{behaupt}{Behauptung}
\newcommand{\gdw}{\Leftrightarrow}

\usepackage{fancyhdr}
\pagestyle{fancy}
\lhead{Stochastik 1 - Gruppe 5 - Blatt Nr. 1}
\rhead{Lennart Braun, John Doe, Kai Robin Sachse}
\cfoot{\thepage}


\title{Stochastik 1 für Informatiker}
\subtitle{Blatt 1 Hausaufgaben (Gruppe 5)}
\author{
	Lennart Braun,
    Foo Bar,
    Foo Bar
}
\date{zum 14. April 2015}

\begin{document}
%\maketitle

\begin{enumerate}[label=\bfseries\arabic*.]
    \item
        \begin{enumerate}[label=(\alph*)]
            \item
                \begin{behaupt}
                    Die Mengengleichung
                    \begin{equation}
                        (A \backslash B) \cup (A \backslash C)
                        = A \backslash (B \cap C)
                    \end{equation}
                    gilt stets.
                \end{behaupt}
                \begin{proof}
                    Um Gleichheit zu zeigen, muss bewiesen werden, dass beide
                    Mengen Teilmengen der jeweils anderen sind.
                    \begin{enumerate}
                        \item
                            $(A \backslash B) \cup (A \backslash C)
                            \subset A \backslash (B \cap C)$ \\
                            \begin{equation}
                                \begin{split}
                                         &x \in (A \backslash B)
                                            \cup (A \backslash C) \\
                                    \gdw &x \in (A \cap B^C) \cup (A \cap C^C) \\
                                    \gdw &x \in A \cap (B^C \cup C^C) \\
                                    \gdw &x \in A \cap (B \cap C)^C \\
                                    \gdw &x \in A \backslash (B \cap C)
                                \end{split}
                            \end{equation}
                            Die Anwendung von mengentheoretischen Umformungen
                            zeigt, dass jedes Element der linken Menge auch in
                            der rechten enthalten sein muss.

                        \item
                            $(A \backslash B) \cup (A \backslash C)
                            \supset A \backslash (B \cap C)$ \\
                            Da ausschließlich Äquivalenzumformungen vorgenommen
                            wurden, gilt obige Argumentation auch in der anderen
                            Richtung.
                    \end{enumerate}
                    Aus i. und ii. folgt die Behauptung.
                \end{proof}

            \item
                \begin{behaupt}
                    Die Mengengleichung
                    \begin{equation}
                        (A \cup B) \backslash C = A \cup (B \backslash C)
                    \end{equation}
                    ist im Allgemeinen nicht gültig.
                \end{behaupt}
                \begin{proof}
                    Seien die Mengen wie folgt definiert
                    \begin{equation}
                        \Omega = A = B = C = \{x\} \text{ .}
                    \end{equation}
                    Das Einsetzen in obige Gleichung führt zum Widerspruch.
                    \begin{equation}
                        \begin{split}
                        (A \cup B) \backslash C &= A \cup (B \backslash C) \\
                        \gdw (\{x\} \cup \{x\}) \backslash \{x\}
                        &= \{x\} \cup (\{x\} \backslash \{x\}) \\
                        \gdw \{x\} \backslash \{x\} &= \{x\} \cup \emptyset \\
                        \gdw \emptyset &= \{x\} \quad \lightning
                        \end{split}
                    \end{equation}
                \end{proof}

        \end{enumerate}

    \item
        \begin{enumerate}[label=(\alph*)]
            \item
                Das Eintreten aller drei Ereignisse $A, B, C$ wird durch
                den folgenden Ausdruck beschrieben:
                \begin{equation}
                    A \cap B \cap C
                \end{equation}

            \item
                Das Eintreten höchstens eines der drei Ereignisse wird durch
                den folgenden Ausdruck beschrieben:
                \begin{equation}
                    (A \cup B \cup C)^C
                    \cup (A \backslash (B \cup C))
                    \cup (B \backslash (A \cup C))
                    \cup (C \backslash (A \cup B))
                \end{equation}

            \item
                Das Eintreten mindestens zwei der drei Ereignisse wird durch
                den folgenden Ausdruck beschrieben:
                \begin{equation}
                    (A \cap B) \cup (A \cap C) \cup (B \cap C)
                \end{equation}

            \item
                Das Eintreten einer ungeraden Anzahl der drei Ereignisse wird
                durch den folgenden Ausdruck beschrieben:
                \begin{equation}
                    (A \cap B \cap C)
                    \cup (A \backslash (B \cup C))
                    \cup (B \backslash (A \cup C))
                    \cup (C \backslash (A \cup B))
                \end{equation}

        \end{enumerate}

    \item
        \begin{enumerate}[label=(\alph*)]
            \item
                Der Wahrscheinlichkeitsraum $(\Omega, P)$ modelliert drei
                aufeinanderfolgenden Münzwürfe.
                Der Grundraum sei definiert als
                \begin{equation}
                    \Omega = \Big\{
                        (x, y, z) \ \vert\  x, y, z \in \{k, z\}
                    \Big\}
                    = \{k, z\}^3
                    \text{ .}
                \end{equation}
                $P$ sei das Laplace-Maß auf $\Omega$, d.\,h. es gelte
                $P(A) = \frac{|A|}{|\Omega|}$ für alle $A \subset \Omega$.

            \item
                Das Ereignis, dass im ersten und im zweiten Wurf jeweils die
                gleiche Münzseite oben liegt, wird durch die Menge $A$
                beschrieben.
                \begin{equation}
                    A = \left\{
                        (x, y, z) \in \Omega \ \vert \ 
                        x = y
                    \right\}
                \end{equation}
                Das Ereignis, dass im zweiten und im dritten Wurf jeweils die
                gleiche Münzseite oben liegt, wird durch die Menge $B$
                beschrieben.
                \begin{equation}
                    B = \left\{
                        (x, y, z) \in \Omega \ \vert \ 
                        y = z
                    \right\}
                \end{equation}

            \item
                $A \cap B$  beschreibt das Ereignis, dass in allen drei
                Münzwürfen die gleiche Seite oben liegt.
                \begin{equation}
                    A \cap B = \left\{
                        (x, y, z) \in \Omega \ \vert \ 
                        x = y = z
                    \right\}
                \end{equation}
                $A \backslash B$ beschreibt das Ereignis, dass in den ersten
                beiden Münzwürfen die gleiche Seite oben liegt und im letzten
                Wurf die andere Seite oben liegt.
                \begin{equation}
                    A \backslash B = \left\{
                        (x, y, z) \in \Omega \ \vert \ 
                        x = y \land y \neq z
                    \right\}
                \end{equation}

            \item
                Die Wahrscheinlichkeit für $A$ beträgt $\frac{1}{2}$, da $A$
                vier Elemente enthält und obige Definition von $P$ gilt.
                \begin{equation}
                    P(A) = \frac{|A|}{|\Omega|} = \frac{4}{8} = \frac{1}{2}
                \end{equation}
                Die Wahrscheinlichkeit für $A \cap B$ beträgt $\frac{1}{4}$,
                da $A \cap B$ zwei Elemente enthält.
                \begin{equation}
                    P(A \cap B) = \frac{|A \cap B|}{|\Omega|}
                    = \frac{2}{8} = \frac{1}{4}
                \end{equation}

        \end{enumerate}

\end{enumerate}

\end{document}
