\documentclass[a4paper]{scrartcl}

% font/encoding packages
\usepackage[utf8]{inputenc}
\usepackage[T1]{fontenc}
\usepackage{lmodern}
\usepackage[ngerman]{babel}

\usepackage{amsmath, amssymb, amsfonts, amsthm}
\allowdisplaybreaks
\usepackage[output-decimal-marker={,}]{siunitx}
\usepackage[shortlabels]{enumitem}
\usepackage[section]{placeins}

\newtheorem*{behaupt}{Behauptung}


\title{Stochastik 1 für Informatiker}
\subtitle{Blatt 1 Hausaufgaben (Gruppe 5)}
\author{
	Lennart Braun,
    Foo Bar,
    Foo Bar
}
\date{zum 14. April 2015}

\begin{document}
\maketitle

\begin{enumerate}[label=\bfseries\arabic*.]
    \item
        \begin{enumerate}[label=(\alph*)]
            \item

            \item

        \end{enumerate}

    \item
        \begin{enumerate}[label=(\alph*)]
            \item
                Das Eintreten aller drei Ereignisse $A, B, C$ wird durch
                den folgenden Ausdruck beschrieben:
                \begin{equation}
                    A \cap B \cap C
                \end{equation}

            \item
                Das Eintreten höchstens eines der drei Ereignisse wird durch
                den folgenden Ausdruck beschrieben:
                \begin{equation}
                    (A \cup B \cup C)^C
                    \cup A \backslash (B \cup C)
                    \cup B \backslash (A \cup C)
                    \cup C \backslash (A \cup B)
                \end{equation}

            \item
                Das Eintreten mindestens zwei der drei Ereignisse wird durch
                den folgenden Ausdruck beschrieben:
                \begin{equation}
                    (A \cap B) \cup (A \cap C) \cup (B \cap C)
                \end{equation}

            \item
                Das Eintreten einer ungeraden Anzahl der drei Ereignisse wird
                durch den folgenden Ausdruck beschrieben:
                \begin{equation}
                    (A \cap B \cap C)
                    \cup A \backslash (B \cup C)
                    \cup B \backslash (A \cup C)
                    \cup C \backslash (A \cup B)
                \end{equation}

        \end{enumerate}

    \item
        \begin{enumerate}[label=(\alph*)]
            \item

            \item

            \item

            \item

        \end{enumerate}

\end{enumerate}

\end{document}
