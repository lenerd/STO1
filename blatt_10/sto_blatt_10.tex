\documentclass[a4paper]{scrartcl}

% font/encoding packages
\usepackage[utf8]{inputenc}
\usepackage[T1]{fontenc}
\usepackage{lmodern}
\usepackage[ngerman]{babel}
\usepackage[ngerman=ngerman-x-latest]{hyphsubst}

\usepackage{amsmath, amssymb, amsfonts, amsthm}
\usepackage{array}
\usepackage{stmaryrd}
\usepackage{marvosym}
\allowdisplaybreaks
\usepackage[output-decimal-marker={,}]{siunitx}
\usepackage[shortlabels]{enumitem}
\usepackage[section]{placeins}
\usepackage{float}
\usepackage{units}
\usepackage{listings}

\newtheorem*{behaupt}{Behauptung}
\newcommand{\gdw}{\Leftrightarrow}
\newcommand{\N}{\mathbb{N}}
\newcommand{\cov}{\operatorname{Cov}}
\newcommand{\e}{\operatorname{E}}
\newcommand{\var}{\operatorname{Var}}
\newcommand{\corr}{\operatorname{Corr}}

\usepackage{fancyhdr}
\pagestyle{fancy}
\lhead{Stochastik 1 - Gruppe 5 - Blatt 10}
\rhead{Lennart Braun, Merlin Koglin, Kai Robin Sachse}
\cfoot{\thepage}


\title{Stochastik 1 für Informatiker}
\subtitle{Blatt 10 Hausaufgaben (Gruppe 5)}
\author{
    Lennart Braun (6523742), \\
    Merlin Koglin (6450362), \\
    Kai Robin Sachse (6450486)
}
\date{zum 30. Juni 2015}

\begin{document}
\maketitle

\begin{enumerate}[label=\bfseries\arabic*.]
    \item
        \begin{enumerate}[label=(\alph*)]
            \item

            \item

        \end{enumerate}

    \item
        \begin{enumerate}[label=(\alph*)]
            \item

            \item

        \end{enumerate}

    \item
        Sei $Y \sim \mathcal{B}_{(n, p)}$ die Anzahl der geworfenen Sechsen in
        $n = 1200$ Würfen mit einem fairen Würfel ($p = \nicefrac{1}{6}$).
        \begin{behaupt}
            Es gilt
            \begin{equation*}
                P\{190 < Y \leq 220\} \approx \num{0,7188} \text{ .}
            \end{equation*}
        \end{behaupt}
        \begin{proof}
            Wir verwenden den zentralen Grenzwertsatz in der Form von Moivre und
            Laplace.
            \begin{equation*}
                \begin{split}
                    P\left\{ 190 < Y \leq 220 \right\}
                    &= P\left\{
                        \underbrace{\frac{190 - np}{\sqrt{np \cdot (1 - p)}}}_a
                        < \frac{Y - np}{\sqrt{np \cdot (1 - p)}} \leq
                        \underbrace{\frac{220 -np}{\sqrt{np \cdot (1 - p)}}}_b
                    \right\} \\
                    &\approx \Phi(b) - \Phi(a) \\
                    &\approx \Phi(\num{1.5492}) - \Phi(\num{-0.7746}) \\
                    &= \Phi(\num{1.5492}) - 1 + \Phi(\num{0.7746}) \\
                    &\approx \num{0,9394} - 1 + \num{0,7794} \\
                    &= \num{0,7188}
                \end{split}
            \end{equation*}
        \end{proof}

\end{enumerate}


\end{document}
