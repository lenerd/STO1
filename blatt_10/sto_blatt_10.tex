\documentclass[a4paper]{scrartcl}

% font/encoding packages
\usepackage[utf8]{inputenc}
\usepackage[T1]{fontenc}
\usepackage{lmodern}
\usepackage[ngerman]{babel}
\usepackage[ngerman=ngerman-x-latest]{hyphsubst}

\usepackage{amsmath, amssymb, amsfonts, amsthm}
\usepackage{array}
\usepackage{stmaryrd}
\usepackage{marvosym}
\allowdisplaybreaks
\usepackage[output-decimal-marker={,}]{siunitx}
\usepackage[shortlabels]{enumitem}
\usepackage[section]{placeins}
\usepackage{float}
\usepackage{units}
\usepackage{listings}

\newtheorem*{behaupt}{Behauptung}
\newcommand{\gdw}{\Leftrightarrow}
\newcommand{\N}{\mathbb{N}}
\newcommand{\cov}{\operatorname{Cov}}
\newcommand{\e}{\operatorname{E}}
\newcommand{\var}{\operatorname{Var}}
\newcommand{\corr}{\operatorname{Corr}}

\usepackage{fancyhdr}
\pagestyle{fancy}
\lhead{Stochastik 1 - Gruppe 5 - Blatt 10}
\rhead{Lennart Braun, Merlin Koglin, Kai Robin Sachse}
\cfoot{\thepage}


\title{Stochastik 1 für Informatiker}
\subtitle{Blatt 10 Hausaufgaben (Gruppe 5)}
\author{
    Lennart Braun (6523742), \\
    Merlin Koglin (6450362), \\
    Kai Robin Sachse (6450486)
}
\date{zum 30. Juni 2015}

\begin{document}
\maketitle

\begin{enumerate}[label=\bfseries\arabic*.]
    \item
        Gebe die Zufallsvariable $X_i$ die Anzahl der Personen an, für deren
        Kommen der Ehemalige $i$ sorgt.
        Es gelten $P\{X_i = 0\} = \num{0,5}$, $P\{X_i = 1\} = \num{0,3}$ und
        $P\{X_i = 2\} = \num{0,2}$ für alle $1 \leq i \leq 100$ sowie
        $Y = \sum_{i=1}^{100} X_i$.
        \begin{enumerate}[label=(\alph*)]
            \item
                \begin{behaupt}
                    Für die Anzahl $Y$ der teilnehmenden Personen gelten
                    \begin{equation*}
                        \e(Y) = 70 \text{ , } \var(Y) = 61 \text{ und }
                        \sigma_Y = \sqrt{61} \text{ .}
                    \end{equation*}
                \end{behaupt}
                \begin{proof}
                    Da alle $X_i$ unabhängig voneinander sind, können wir die
                    Größen wie folgt bestimmen.

                    Erwartungswert:
                    \begin{equation*}
                        \begin{split}
                            \e(Y) &= \e\left( \sum_{i=1}^{100} X_i \right) \\
                                  &= \sum_{i=1}^{100} \e(X_i) \\
                                  &= \sum_{i=1}^{100} (0 \cdot \num{0,5} + 1
                                     \cdot \num{0,3} + 2 \cdot \num{0,2}) \\
                                  &= 100 \cdot \num{0,7} = 70
                        \end{split}
                    \end{equation*}
                    Varianz:
                    \begin{equation*}
                        \begin{split}
                            \var(Y)
                            &= \var\left( \sum_{i=1}^{100} X_i \right) \\
                            &= \sum_{i=1}^{100} \var(X_i) \\
                            &= \sum_{i=1}^{100}
                               \Big( \e(X_i^2) - (\e(X_i))^2 \Big) \\
                            &= \sum_{i=1}^{100}
                               \Big( \num{1,1} - \num{0,49} \Big) \\
                            &= 100 \cdot \num{0,61} = 61
                        \end{split}
                    \end{equation*}
                    Standardabweichung:
                    \begin{equation*}
                        \sigma_Y = \sqrt{\var{Y}}
                        = \sqrt{61} \approx \num{7,8102}
                    \end{equation*}
                \end{proof}

            \item
                \begin{behaupt}
                    Die Wahrscheinlichkeit, dass mehr als 80 Leute kommen,
                    beträgt ohne Stetigkeitskorrektur ca. \num{0,1003} und mit
                    \num{0,0901}.
                \end{behaupt}
                \begin{proof}
                    Wir wenden den zentralen Grenzwertsatz an.
                    Es seien $\mu = \e(X_i)$ und $\sigma^2 = \var(X_i)$ für
                    $1 \leq i \leq 100$.
                    \begin{equation*}
                        \begin{split}
                            P\{Y > 80\} &= 1 - P\{Y \leq 80\} \\
                            &= 1 - P \left\lbrace
                            \frac{Y - n \cdot \mu}{\sqrt{n \cdot \sigma^2}}
                            \leq
                            \frac{80 - n \cdot \mu}{\sqrt{n\cdot \sigma^2}}
                            \right\rbrace \\
                            &\approx 1 - \Phi \left(
                            \frac{80 - 100 \cdot 0,7}{\sqrt{100 \cdot 0,61}}
                            \right) \\
                            &= 1 - \Phi \left( \frac{10}{\sqrt{61}} \right) \\
                            &\approx 1 - \Phi(\num{1,2804})
                            \approx 1 - \num{0,8997} = \num{0,1003}
                        \end{split}
                    \end{equation*}
                    % \begin{equation*}
                    %     \begin{split}
                    %         P\{Y>80\}&=P\{Y> 79,5\}\\
                    %         &= 1-P\{Y\le 79,5\}\\
                    %         &=1-P\left\{\frac{Y-n\cdot E(X_1)}{\sqrt{n\cdot Var X_1}}
                    %         \le \frac{79,5-n\cdot E(X_1)}{\sqrt{n\cdot Var X_1}}\right\} \\
                    %         &=1-\Phi\left(\frac{79,5-100\cdot 0,7}{\sqrt{100\cdot 0,61}}\right)\\
                    %         &\approx 1-\Phi\left(\frac{9,5}{\sqrt{61}}\right)\\
                    %         &\approx 1-\Phi(1,2163)\approx 1-0,8869=0,1131
                    %     \end{split}
                    % \end{equation*}
                    \begin{equation*}
                        \begin{split}
                            P\{Y > 80\} &= P\{Y > \num{80,5}\} \\
                            &= 1 - P\{Y \leq \num{80,5}\} \\
                            &= 1 - P \left\lbrace
                            \frac{Y - n \cdot \mu}{\sqrt{n \cdot \sigma^2}}
                            \leq
                            \frac{\num{80,5} - n \cdot \mu}
                                 {\sqrt{n\cdot \sigma^2}}
                            \right\rbrace \\
                            &\approx 1 - \Phi \left(
                            \frac{\num{80,5} - 100 \cdot 0,7}{\sqrt{100 \cdot 0,61}}
                            \right) \\
                            &= 1 - \Phi \left( \frac{\num{10,5}}
                                                    {\sqrt{61}} \right) \\
                            &\approx 1 - \Phi(\num{1,3444})
                            \approx 1 - \num{0,9099} = \num{0,0901}
                        \end{split}
                    \end{equation*}
                \end{proof}
            

        \end{enumerate}

    \item
        \begin{enumerate}[label=(\alph*)]
            \item
                Angenommen die Geschlechter aller Babys sind unabhängig und die
                Wahrscheinlichkeit für ein Mädchen betrage $p = \num{0,5}$.
                Sei $Y \sim \mathcal{B}_{(n, p)}$ eine Zufallsvariable, welche
                die Anzahl der Mädchen in $n = 1290$ Geburten angibt.
                \begin{equation*}
                    \begin{split}
                        P\{Y \leq 621\}
                        &= P\left\{ \frac{Y - np}{\sqrt{np(1-p)}} \leq
                            \frac{621 - np}{\sqrt{np(1-p)}} \right\} \\
                        &\approx \Phi\left(
                            \frac{621 - 1290 \cdot \num{0,5}}
                            {\sqrt{1290 \cdot \num{0,5} \cdot \num{0,5}}}
                        \right) \\
                        &= \Phi\left( \frac{-24}{\sqrt{\num{322,5}}} \right) \\
                        &\approx \Phi(\num{-1,3364})
                        = 1 - \Phi(\num{1,3364})
                        \approx 1 - \num{0,9099}
                        = \num{0,0901}
                    \end{split}
                \end{equation*}

            %\item Die W. das bei einer 50/50 W. von Jungen und Mädchen, 621 Mädchen geboren werden liegt bei unter 10 Prozent. Es liegt also nahe, das die Annahme, dass die W. für ein Madchen wirklich $0,5$ beträgt, falsch ist. 

            % Ich habe das mal mit dem Beispiel aus der Vorlesung gerechnet:
            %Da die Aufgabe nur 1P gibt ging ich davon aus sie wollten nur eine heuristische Analyse
            % Auf die Punkte hatte ich gar nicht geachtet.
            % Bin aber zu einem anderen Ergebnis gekommen.
            \item
                Wir wählen $\alpha = \num{0,05}$ und dann ein $\varepsilon > 0$,
                so dass die Wahrscheinlichkeit, dass die Anzahl an Mädchen bei
                fairen Geburten um mindestens $\varepsilon$ vom Erwartungswert
                abweicht, höchstens $\alpha$ ist.
                \begin{equation*}
                    P \left\{
                        \left| \frac{1}{n} (Y - np) \right| \geq \varepsilon
                    \right\} \leq \alpha
                \end{equation*}
                Mit Hilfe des zentralen Grenzwertsatzes können wir dieses
                $\varepsilon$ bestimmen.
                \begin{equation*}
                    \begin{alignedat}{2}
                        && P \left\{
                            \left| \frac{1}{n} (Y - np) \right| \geq \varepsilon
                        \right\} 
                        \leq \alpha \\
                        \gdw&& P \left\{
                            \left| \frac{Y - np}{\sqrt{np(1-p)}} \right|
                            \geq 2 \sqrt{n} \varepsilon
                        \right\} 
                        \leq \alpha \\
                        \gdw&& P \left\{
                            \frac{Y - np}{\sqrt{np(1-p)}}
                            \geq 2 \sqrt{n} \varepsilon
                            \text{ oder }
                            -\frac{Y - np}{\sqrt{np(1-p)}}
                            \geq 2 \sqrt{n} \varepsilon
                        \right\} 
                        \leq \alpha \\
                        \gdw&& P \left\{
                            \frac{Y - np}{\sqrt{np(1-p)}}
                            \geq 2 \sqrt{n} \varepsilon
                        \right\} 
                        +
                        P \left\{
                            \frac{Y - np}{\sqrt{np(1-p)}}
                            \leq -2 \sqrt{n} \varepsilon
                        \right\} 
                        \leq \alpha \\
                        \gdw&&
                        1 - \Phi(2 \sqrt{n} \varepsilon)
                        +
                        \Phi(-2 \sqrt{n} \varepsilon)
                        \lessapprox \alpha \\
                        \gdw&&
                        2 - 2\Phi(2 \sqrt{n} \varepsilon)
                        \lessapprox \alpha \\
                        \gdw&&
                        \frac{\Phi^{-1} \left( 1 - \frac{\alpha}{2} \right)}
                             {2\sqrt{n}}
                        \lessapprox \varepsilon \\
                        \gdw&&
                        \frac{\Phi^{-1}(\num{0,975})}
                             {2\sqrt{1290}}
                        \lessapprox \varepsilon \\
                        \gdw&&
                        \frac{\num{1,96}}
                             {2\sqrt{1290}}
                        \lessapprox \varepsilon \\
                        \gdw&&
                        \num{0.0273}
                        \lessapprox \varepsilon \\
                    \end{alignedat}
                \end{equation*}
                Wir sehen die Geburten als unfair an, wenn die relative
                Häufigkeit von Mädchen außerhalb von
                $(\num{0,5} - \varepsilon, \num{0,5} + \varepsilon) =
                (\num{0,4727},\ \num{0,5273})$ liegt.
                Dies ist mit $\nicefrac{621}{1290} \approx \num{0,4814}$ nicht der
                Fall.

        \end{enumerate}

    \item
        Sei $Y \sim \mathcal{B}_{(n, p)}$ die Anzahl der geworfenen Sechsen in
        $n = 1200$ Würfen mit einem fairen Würfel ($p = \nicefrac{1}{6}$).
        \begin{behaupt}
            Es gilt
            \begin{equation*}
                P\{190 < Y \leq 220\} \approx \num{0,7188} \text{ .}
            \end{equation*}
        \end{behaupt}
        \begin{proof}
            Wir verwenden den zentralen Grenzwertsatz in der Form von Moivre und
            Laplace.
            \begin{equation*}
                \begin{split}
                    P\left\{ 190 < Y \leq 220 \right\}
                    &= P\left\{
                        \underbrace{\frac{190 - np}{\sqrt{np \cdot (1 - p)}}}_a
                        < \frac{Y - np}{\sqrt{np \cdot (1 - p)}} \leq
                        \underbrace{\frac{220 -np}{\sqrt{np \cdot (1 - p)}}}_b
                    \right\} \\
                    &\approx \Phi(b) - \Phi(a) \\
                    &\approx \Phi(\num{1.5492}) - \Phi(\num{-0.7746}) \\
                    &= \Phi(\num{1.5492}) - 1 + \Phi(\num{0.7746}) \\
% also in der Tabelle die ich hab steht phi(1,54)=0,9382 ??                    
                    &\approx \num{0,9394} - 1 + \num{0,7794} \\
                    &= \num{0,7188}
                \end{split}
            \end{equation*}
        \end{proof}

\end{enumerate}


\end{document}
