\documentclass[a4paper]{scrartcl}

% font/encoding packages
\usepackage[utf8]{inputenc}
\usepackage[T1]{fontenc}
\usepackage{lmodern}
\usepackage[ngerman]{babel}
\usepackage[ngerman=ngerman-x-latest]{hyphsubst}

\usepackage{amsmath, amssymb, amsfonts, amsthm}
\usepackage{array}
\usepackage{stmaryrd}
\usepackage{marvosym}
\allowdisplaybreaks
\usepackage[output-decimal-marker={,}]{siunitx}
\usepackage[shortlabels]{enumitem}
\usepackage[section]{placeins}
\usepackage{float}
\usepackage{units}
\usepackage{listings}

\newtheorem*{behaupt}{Behauptung}
\newcommand{\gdw}{\Leftrightarrow}
\newcommand{\N}{\mathbb{N}}
\newcommand{\cov}{\operatorname{Cov}}
\newcommand{\e}{\operatorname{E}}
\newcommand{\var}{\operatorname{Var}}
\newcommand{\corr}{\operatorname{Corr}}

\usepackage{fancyhdr}
\pagestyle{fancy}
\lhead{Stochastik 1 - Gruppe 5 - Blatt 10}
\rhead{Lennart Braun, Merlin Koglin, Kai Robin Sachse}
\cfoot{\thepage}


\title{Stochastik 1 für Informatiker}
\subtitle{Blatt 10 Hausaufgaben (Gruppe 5)}
\author{
    Lennart Braun (6523742), \\
    Merlin Koglin (6450362), \\
    Kai Robin Sachse (6450486)
}
\date{zum 30. Juni 2015}

\begin{document}
\maketitle

\begin{enumerate}[label=\bfseries\arabic*.]
    \item
        Gebe die Zufallsvariable $X_i$ die Anzahl der Personen an, für deren
        Kommen der Ehemalige $i$ sorgt.
        Es gelten $P\{X_i = 0\} = \num{0,5}$, $P\{X_i = 1\} = \num{0,3}$ und
        $P\{X_i = 2\} = \num{0,2}$ für alle $1 \leq i \leq 100$ sowie
        $Y = \sum_{i=1}^{100} X_i$.
        \begin{enumerate}[label=(\alph*)]
            \item
                \begin{behaupt}
                    Für die Anzahl $Y$ der teilnehmenden Personen gelten
                    \begin{equation*}
                        \e(Y) = 70 \text{ , } \var(Y) = 61 \text{ und }
                        \sigma_Y = \sqrt{61} \text{ .}
                    \end{equation*}
                \end{behaupt}
                \begin{proof}
                    Da alle $X_i$ unabhängig voneinander sind, können wir die
                    Größen wie folgt bestimmen.

                    Erwartungswert:
                    \begin{equation*}
                        \begin{split}
                            \e(Y) &= \e\left( \sum_{i=1}^{100} X_i \right) \\
                                  &= \sum_{i=1}^{100} \e(X_i) \\
                                  &= \sum_{i=1}^{100} (0 \cdot \num{0,5} + 1
                                     \cdot \num{0,3} + 2 \cdot \num{0,2}) \\
                                  &= 100 \cdot \num{0,7} = 70
                        \end{split}
                    \end{equation*}
                    Varianz:
                    \begin{equation*}
                        \begin{split}
                            \var(Y)
                            &= \var\left( \sum_{i=1}^{100} X_i \right) \\
                            &= \sum_{i=1}^{100} \var(X_i) \\
                            &= \sum_{i=1}^{100}
                               \Big( \e(X_i^2) - (\e(X_i))^2 \Big) \\
                            &= \sum_{i=1}^{100}
                               \Big( \num{1,1} - \num{0,49} \Big) \\
                            &= 100 \cdot \num{0,61} = 61
                        \end{split}
                    \end{equation*}
                    Standardabweichung:
                    \begin{equation*}
                        \sigma_Y = \sqrt{\var{Y}}
                        = \sqrt{61} \approx \num{7,8102}
                    \end{equation*}
                \end{proof}

            \item \begin{behaupt} Die W. das mehr als 80 Leute kommen beträgt BLUBB
            \end{behaupt}
            
            \begin{proof}
            Mit ZGWS gilt ohne Stetigkeitsapprox
            \begin{align*}
            P\{Y>80\}&=1-P\{Y\le 80\}=1-P\{\frac{Y-n\cdot E(X_1)}{\sqrt{n\cdot Var X_1}}\le \frac{80-n\cdot E(X_1)}{\sqrt{n\cdot Var X_1}}\} \\
            &\approx \Phi\{\frac{80-100\cdot 0,7}{\sqrt{100\cdot 0,61}}\}= 1-\Phi(\frac{10}{\sqrt{61}})\\
            &\approx 1-\Phi(1,2804)\approx 1-0,8997=0,1003
            \end{align*}
            Mit Korrektur
            \begin{align*}
            P\{Y>80\}&=P\{Y> 79,5\}= 1-P\{Y\le 79,5\}=1-P\{\frac{Y-n\cdot E(X_1)}{\sqrt{n\cdot Var X_1}}\le \frac{79,5-n\cdot E(X_1)}{\sqrt{n\cdot Var X_1}}\} \\
            &=1-P\{\frac{79,5-100\cdot 0,7}{\sqrt{100\cdot 0,61}}\}\approx 1-\Phi(\frac{9,5}{\sqrt{61}})\\
            &\approx 1-\Phi(1,2163)\approx 1-0,8869=0,1131
            \end{align*}
            \end{proof}
            

        \end{enumerate}

    \item
        \begin{enumerate}[label=(\alph*)]
            \item

            \item

        \end{enumerate}

    \item
        Sei $Y \sim \mathcal{B}_{(n, p)}$ die Anzahl der geworfenen Sechsen in
        $n = 1200$ Würfen mit einem fairen Würfel ($p = \nicefrac{1}{6}$).
        \begin{behaupt}
            Es gilt
            \begin{equation*}
                P\{190 < Y \leq 220\} \approx \num{0,7188} \text{ .}
            \end{equation*}
        \end{behaupt}
        \begin{proof}
            Wir verwenden den zentralen Grenzwertsatz in der Form von Moivre und
            Laplace.
            \begin{equation*}
                \begin{split}
                    P\left\{ 190 < Y \leq 220 \right\}
                    &= P\left\{
                        \underbrace{\frac{190 - np}{\sqrt{np \cdot (1 - p)}}}_a
                        < \frac{Y - np}{\sqrt{np \cdot (1 - p)}} \leq
                        \underbrace{\frac{220 -np}{\sqrt{np \cdot (1 - p)}}}_b
                    \right\} \\
                    &\approx \Phi(b) - \Phi(a) \\
                    &\approx \Phi(\num{1.5492}) - \Phi(\num{-0.7746}) \\
                    &= \Phi(\num{1.5492}) - 1 + \Phi(\num{0.7746}) \\
                    &\approx \num{0,9394} - 1 + \num{0,7794} \\
                    &= \num{0,7188}
                \end{split}
            \end{equation*}
        \end{proof}

\end{enumerate}


\end{document}
