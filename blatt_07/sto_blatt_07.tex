\documentclass[a4paper]{scrartcl}

% font/encoding packages
\usepackage[utf8]{inputenc}
\usepackage[T1]{fontenc}
\usepackage{lmodern}
\usepackage[ngerman]{babel}
\usepackage[ngerman=ngerman-x-latest]{hyphsubst}

\usepackage{amsmath, amssymb, amsfonts, amsthm}
\usepackage{array}
\usepackage{stmaryrd}
\usepackage{marvosym}
\allowdisplaybreaks
\usepackage[output-decimal-marker={,}]{siunitx}
\usepackage[shortlabels]{enumitem}
\usepackage[section]{placeins}
\usepackage{float}
\usepackage{units}

\newtheorem*{behaupt}{Behauptung}
\newcommand{\gdw}{\Leftrightarrow}
\newcommand{\N}{\mathbb{N}}

\usepackage{fancyhdr}
\pagestyle{fancy}
\lhead{Stochastik 1 - Gruppe 5 - Blatt 7}
\rhead{Lennart Braun, Merlin Koglin, Kai Robin Sachse}
\cfoot{\thepage}


\title{Stochastik 1 für Informatiker}
\subtitle{Blatt 7 Hausaufgaben (Gruppe 5)}
\author{
    Lennart Braun (6523742), \\
    Merlin Koglin (6450362), \\
    Kai Robin Sachse (6450486)
}
\date{zum 9. Juni 2015}

\begin{document}
\maketitle

\begin{enumerate}[label=\bfseries\arabic*.]
    \item
        Im Erdgeschoss steigen 6 Personen ein und wählen unabhängig voneinander
        eins von 8 Stockwerken aus.
        \begin{behaupt}
            Die erwartete Anahl von Stockwerken, bei denen mindestens eine der
            Personen aussteigt beträgt ca. \num{4,410}.
        \end{behaupt}
        \begin{proof}
            Wir betrachten den Wahrscheinlichkeitsraum $(\Omega, P)$ mit dem
            Laplace-Maß $P$ und
            \begin{equation*}
                \Omega = \Big\{ (\omega_1, \ldots, \omega_6)
                                \in \{1, \ldots, 8\} \Big\}
                \text{ .}
            \end{equation*}
            Für $\omega \in \Omega$ bedeutet $\omega_i = k$, dass Person $i$ im
            Stockwerk $k$ aussteigt.
            Im Folgenden gelte $k \in \{1, \ldots, 8\}$.
            Sei $A_k$ das Ereigniss, dass mindestens eine Person im Stockwerk
            $k$ aussteigt.
            Die Wahrscheinlichkeit $P(A_k)$ lässt sich einfach über das
            Komplement $A_k^C$, dem Ereignis, dass keine Person im Stockwerk $k$
            aussteigt, berechnen.
            Diese beträgt offensichtlich $P(A_k^C) = \left(\frac{7}{8}\right)^6$
            Damit gilt $P(A_k) = 1 - \left( \frac{7}{8} \right)^6$.
            Wir führen nun die Indikatorvariablen $X_k \in \{0, 1\}$ ein welche
            mit der Wahrscheinlichkeit $P(A_k)$ den Wert $1$ annimmt.
            Sei $X$ die Summe über alle $X_k$, welche angibt, in wie vielen
            verschiedenen Stockwerken Personen aussteigen.
            Wir bestimmen nun den Erwartungswert von $X$.
            \begin{equation*}
                \begin{split}
                    E(X) &= E \left( \sum_{k = 1}^8 X_k \right) \\
                         &= \sum_{k = 1}^8 E(X_k) \\
                         &= \sum_{k = 1}^8 0 \cdot P(A_k^C) + 1 \cdot P(A_k) \\
                         &= \sum_{k = 1}^8 1 - \left( \frac{7}{8} \right)^6 \\
                         &= 8 \cdot
                            \left( 1 - \left( \frac{7}{8} \right)^6 \right) \\
                         &\approx \num{4,410}
                \end{split}
            \end{equation*}
        \end{proof}

    \item
        Es sei $X \sim \mathcal{H}_{(N, M, n)}$ eine Zufallsvariable auf einem
        diskreten Wahrscheinlichkeitsraum $(\Omega, P)$.
        \begin{behaupt}
            Es gilt
            \begin{equation*}
                E(X) = n \cdot \frac{M}{N}
            \end{equation*}
        \end{behaupt}
        \begin{proof}
            \begin{equation*}
                \begin{split}
                    E(X)
                    &= \sum_{m=0}^n m \cdot \mathcal{H}_{(N, M, n)} (\{m\}) \\
                    &= \sum_{m=0}^n m \cdot
                        \frac{\binom{M}{m} \cdot \binom{N-M}{n-m}}
                             {\binom{N}{n}} \\
                    &= \sum_{m=1}^n m \cdot
                        \frac{\binom{M}{m} \cdot \binom{N-M}{n-m}}
                             {\binom{N}{n}} \\
                    &= \sum_{m=1}^n m \cdot
                        \frac{\frac{M}{m} \cdot \binom{M-1}{m-1} \cdot
                        \binom{N-M}{n-m}}{\frac{N}{n} \cdot \binom{N-1}{n-1}} \\
                    &= n \cdot \frac{M}{N} \cdot \sum_{m=1}^n
                        \frac{\binom{M-1}{m-1} \cdot
                        \binom{N-M}{n-m}}{\binom{N-1}{n-1}} \\
                    &= n \cdot \frac{M}{N} \cdot \sum_{m=0}^{n-1}
                        \frac{\binom{M-1}{m} \cdot
                        \binom{N-M}{(n-1) -m}}{\binom{N-1}{n-1}} \\
                    &= n \cdot \frac{M}{N} \cdot \underbrace{\sum_{m=0}^{n-1}
                        \mathcal{H}_{(N-1, M-1, n-1)}(\{m\})}_{= 1} \\
                    &= n \cdot \frac{M}{N}
                \end{split}
            \end{equation*}
        \end{proof}

    \item
        Die Zufallsvariable $X\colon \Omega \to \{0, 1, 2, 3\}$ aus einem
        diskreten Wahrscheinlichkeitsraum $(\Omega, P)$ gibt die Anzahl der Tore
        an, die ein Stürmer schießt.
        Im Erwartungswert schießt der Stürmer ein Tor pro Spiel.
        Die Wahrscheinlichkeit ein Tor zu schießen ist dreimal so groß wie die,
        zwei Tore zu schießen.
        Die Wahrscheinlichkeit kein Tor zu schießen ist 15-mal so groß wie die,
        drei Tore zu schießen.
        \begin{behaupt}
            Für die Wahrscheinlichkeiten $p_i = P\{X = i\}$ für
            $i \in \{0, 1, 2, 3\}$ gilt folgendes.
            \begin{align*}
                \begin{split}
                    p_0 &= \frac{15}{68} \qquad
                    p_1 = \frac{39}{68} \\
                    p_2 &= \frac{13}{68} \qquad
                    p_3 = \frac{1}{68}
                \end{split}
            \end{align*}
        \end{behaupt}
        \begin{proof}
            Aus den gegebenen Informationen können ein Gleichungssystem
            aufstellen.
            Die erste Gleichung muss gelten, da die Summe der
            Wahrscheinlichkeiten aller Ereignisse 1 ergibt.
            Die zweite Gleichung ergibt sich aus der Definition des
            Erwartungswertes: $E(X) = \sum_{x \in X(\Omega)} x \cdot P\{X =x\}$.
            Die restlichen zwei Gleichungen sind die bekannten Verhältnisse der
            Wahrscheinlichkeiten $p_1$ und $p_2$ bzw. $p_0$ und $p_3$.
            \begin{equation*}
                \begin{array}{crcrcrcr @{{}={}} r}
                      & p_0 & + & p_1 & + &  p_2 & + &   p_3 & 1 \\
                      &     & + & p_1 & + & 2p_2 & + &  3p_3 & 1 \\
                      &     & - & p_1 & + & 3p_2 &   &       & 0 \\
                    - & p_0 &   &     &   &      & + & 15p_3 & 0 \\
                \end{array}
            \end{equation*}
            Wir lösen das Gleichungssystem mit dem Gauss-Algorithmus.
            \begin{equation*}
                \begin{array}{crcrcrcr @{{}={}} r}
                      & p_0 & + & p_1 & + &  p_2 & + &   p_3 & 1 \\
                      &     & + & p_1 & + & 2p_2 & + &  3p_3 & 1 \\
                      &     & - & p_1 & + & 3p_2 &   &       & 0 \\
                      &     & + & p_1 & + &  p_2 & + & 16p_3 & 1 \\
                \end{array}
            \end{equation*}
            \begin{equation*}
                \begin{array}{crcrcrcr @{{}={}} r}
                      & p_0 &&     & - &  p_2 & - &  2p_3 & 0 \\
                      &     && p_1 & + & 2p_2 & + &  3p_3 & 1 \\
                      &     &&     &   & 5p_2 & + &  3p_3 & 1 \\
                      &     &&     & - &  p_2 & + & 13p_3 & 0 \\
                \end{array}
            \end{equation*}
            \begin{equation*}
                \setlength{\extrarowheight}{3pt}
                \begin{array}{crcrcrcr @{{}={}} r}
                    & p_0 &&     &&      & - &  \frac{7}{5}p_3 & \frac{1}{5} \\
                    &     && p_1 &&      & + &  \frac{9}{5}p_3 & \frac{3}{5} \\
                    &     &&     &&  p_2 & + &  \frac{3}{5}p_3 & \frac{1}{5} \\
                    &     &&     &&      &   & \frac{68}{5}p_3 & \frac{1}{5} \\
                \end{array}
            \end{equation*}
            \begin{equation*}
                \setlength{\extrarowheight}{3pt}
                \begin{array}{crcrcrcr @{{}={}} r}
                    & p_0 &&     &&      &&     & \frac{15}{68} \\
                    &     && p_1 &&      &&     & \frac{39}{68} \\
                    &     &&     &&  p_2 &&     & \frac{13}{68} \\
                    &     &&     &&      && p_3 &  \frac{1}{68} \\
                \end{array}
            \end{equation*}
            Es ist leicht zu sehen, dass die Verhältnisse zwischen $p_1$ und
            $p_2$ bzw. $p_0$ und $p_3$ stimmen und die Summe aller
            Wahrscheinlichkeiten 1 ergibt.
            Der Erwartungswert stimmt ebenfalls mit dem geforderten überein:
            \begin{equation*}
                E(X)
                = 0 \cdot \frac{15}{68} + 1 \cdot \frac{39}{68}
                + 2 \cdot \frac{13}{68} + 3 \cdot \frac{1}{68}
                = \frac{39 + 26 + 3}{68} = \frac{68}{68} = 1
                \qedhere
            \end{equation*}
        \end{proof}

\end{enumerate}


\end{document}
