\documentclass[a4paper]{scrartcl}

% font/encoding packages
\usepackage[utf8]{inputenc}
\usepackage[T1]{fontenc}
\usepackage{lmodern}
\usepackage[ngerman]{babel}
\usepackage[ngerman=ngerman-x-latest]{hyphsubst}

\usepackage{amsmath, amssymb, amsfonts, amsthm}
\usepackage{stmaryrd}
\usepackage{marvosym}
\allowdisplaybreaks
\usepackage[output-decimal-marker={,}]{siunitx}
\usepackage[shortlabels]{enumitem}
\usepackage[section]{placeins}
\usepackage{float}
\usepackage{units}

\newtheorem*{behaupt}{Behauptung}
\newcommand{\gdw}{\Leftrightarrow}
\newcommand{\N}{\mathbb{N}}

\usepackage{fancyhdr}
\pagestyle{fancy}
\lhead{Stochastik 1 - Gruppe 5 - Blatt 7}
\rhead{Lennart Braun, Merlin Koglin, Kai Robin Sachse}
\cfoot{\thepage}


\title{Stochastik 1 für Informatiker}
\subtitle{Blatt 7 Hausaufgaben (Gruppe 5)}
\author{
    Lennart Braun (6523742), \\
    Merlin Koglin (6450362), \\
    Kai Robin Sachse (6450486)
}
\date{zum 9. Juni 2015}

\begin{document}
\maketitle

\begin{enumerate}[label=\bfseries\arabic*.]
    \item

    \item
        Es sei $X \sim \mathcal{H}_{(N, M, n)}$ eine Zufallsvariable auf einem
        diskreten Wahrscheinlichkeitsraum $(\Omega, P)$.
        \begin{behaupt}
            Es gilt
            \begin{equation*}
                E(X) = n \cdot \frac{M}{N}
            \end{equation*}
        \end{behaupt}
        \begin{proof}
            \begin{equation*}
                \begin{split}
                    E(X)
                    &= \sum_{m=0}^n m \cdot \mathcal{H}_{(N, M, n)} (\{m\}) \\
                    &= \sum_{m=0}^n m \cdot
                        \frac{\binom{M}{m} \cdot \binom{N-M}{n-m}}
                             {\binom{N}{n}} \\
                    &= \sum_{m=1}^n m \cdot
                        \frac{\binom{M}{m} \cdot \binom{N-M}{n-m}}
                             {\binom{N}{n}} \\
                    &= \sum_{m=1}^n m \cdot
                        \frac{\frac{M}{m} \cdot \binom{M-1}{m-1} \cdot
                        \binom{N-M}{n-m}}{\frac{N}{n} \cdot \binom{N-1}{n-1}} \\
                    &= n \cdot \frac{M}{N} \cdot \sum_{m=1}^n
                        \frac{\binom{M-1}{m-1} \cdot
                        \binom{N-M}{n-m}}{\binom{N-1}{n-1}} \\
                    &= n \cdot \frac{M}{N} \cdot \sum_{m=0}^{n-1}
                        \frac{\binom{M-1}{m} \cdot
                        \binom{N-M}{(n-1) -m}}{\binom{N-1}{n-1}} \\
                    &= n \cdot \frac{M}{N} \cdot \underbrace{\sum_{m=0}^{n-1}
                        \mathcal{H}_{(N-1, M-1, n-1)}(\{m\})}_{= 1} \\
                    &= n \cdot \frac{M}{N}
                \end{split}
            \end{equation*}
        \end{proof}

    \item

\end{enumerate}


\end{document}
