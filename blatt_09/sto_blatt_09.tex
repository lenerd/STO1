\documentclass[a4paper]{scrartcl}

% font/encoding packages
\usepackage[utf8]{inputenc}
\usepackage[T1]{fontenc}
\usepackage{lmodern}
\usepackage[ngerman]{babel}
\usepackage[ngerman=ngerman-x-latest]{hyphsubst}

\usepackage{amsmath, amssymb, amsfonts, amsthm}
\usepackage{array}
\usepackage{stmaryrd}
\usepackage{marvosym}
\allowdisplaybreaks
\usepackage[output-decimal-marker={,}]{siunitx}
\usepackage[shortlabels]{enumitem}
\usepackage[section]{placeins}
\usepackage{float}
\usepackage{units}
\usepackage{listings}

\newtheorem*{behaupt}{Behauptung}
\newcommand{\gdw}{\Leftrightarrow}
\newcommand{\N}{\mathbb{N}}
\newcommand{\cov}{\operatorname{Cov}}
\newcommand{\e}{\operatorname{E}}
\newcommand{\var}{\operatorname{Var}}
\newcommand{\corr}{\operatorname{Corr}}

\usepackage{fancyhdr}
\pagestyle{fancy}
\lhead{Stochastik 1 - Gruppe 5 - Blatt 9}
\rhead{Lennart Braun, Merlin Koglin, Kai Robin Sachse}
\cfoot{\thepage}


\title{Stochastik 1 für Informatiker}
\subtitle{Blatt 9 Hausaufgaben (Gruppe 5)}
\author{
    Lennart Braun (6523742), \\
    Merlin Koglin (6450362), \\
    Kai Robin Sachse (6450486)
}
\date{zum 23. Juni 2015}

\begin{document}
\maketitle

\begin{enumerate}[label=\bfseries\arabic*.]
    \item
        Sei $X$ die Anzahl an Zugriffen pro Zeiteinheit auf einen Server.
        Und es gelten $\e(X) = 400$ und $\var(X) = 400$.
        \begin{behaupt}
            Mit der Markov-Ungleichung lässt sich die Wahrscheinlichkeit für
            mindestens 1200 Zugriffe in der nächsten Zeiteinheit abschätzen als:
            \begin{equation*}
                P\{X \geq 1200\} \leq \frac{1}{3}
            \end{equation*}
        \end{behaupt}
        \begin{proof}
            Da $X \geq 0$ gilt $X = |X|$.
            In die Markov-Ungleichung eingesetzt, ergibt sich folgendes.
            \begin{equation*}
                P\{X \geq 1200\} = P\{|X| \geq 1200\}
                \leq \frac{\e(X)}{1200} = \frac{400}{1200} = \frac{1}{3}
            \end{equation*}
        \end{proof}

        \begin{behaupt}
            Mit der Chebyshev-Ungleichung lässt sich die Wahrscheinlichkeit für
            mindestens 1200 Zugriffe in der nächsten Zeiteinheit abschätzen als:
            \begin{equation*}
                P\{X \geq 1200\} \leq \frac{1}{1600}
            \end{equation*}
        \end{behaupt}
        \begin{proof}
            \begin{equation*}
                \begin{alignedat}{2}
                    P\{X \geq 1200\} &= P\{X - \e(X) \geq 800\} \\
                                     &\leq P\{|X - \e(X)| \geq 800\}
                    \qquad &\text{da } X - \e(X) \leq |X - \e(X)| \\
                    &\leq \frac{\var(X)}{800^2}
                    \qquad &\text{Chebyshev's Ungleichung} \\
                    &= \frac{400}{800^2} = \frac{1}{1600}
                \end{alignedat}
            \end{equation*}
        \end{proof}

    \item
        Ein fairer Würfel wird $n$-mal unabhängig voneinander geworfen, $X_i$
        sei die Augenzahl des $i$-ten Wurfes für $1 \leq i \leq n$.
        \begin{enumerate}[label=(\alph*)]
            \item
                \begin{behaupt}
                    Es gelten
                    \begin{equation*}
                        \e \left( \frac{1}{n} \cdot \sum_{i=0}^n X_i \right)
                        = \num{3,5}
                        \quad \text{ und } \quad
                        \var \left( \frac{1}{n} \cdot \sum_{i=0}^n X_i \right)
                        = \frac{35}{12 \cdot n}
                        \text{ .}
                    \end{equation*}
                \end{behaupt}
                \begin{proof}
                    Wir berechnen zunächst den Erwartungswert und die Varianz
                    eines einzelnen Wurfes $X_i$.
                    \begin{equation*}
                        \e(X_i) = \frac{1}{6} \cdot \sum_{k=1}^6 k = \frac{7}{2}
                    \end{equation*}
                    \begin{equation*}
                        \begin{split}
                            \var(X_i) 
                            &= \e(X_i^2) - (\e(X_i))^2 \\
                            &= \frac{1}{6} \cdot \sum_{k=1}^6 k^2
                                - \left( \frac{7}{2} \right)^2 \\
                            &= \frac{35}{12}
                        \end{split}
                    \end{equation*}

                    Für die Berechnung des Erwartungswertes des arithmetischen
                    Mittels der Würfe verwenden wir die Linearität des
                    Erwartungswertes und das oben berechnete $\e(X_i)$.
                    \begin{equation*}
                        \begin{split}
                            \e \left( \frac{1}{n} \cdot \sum_{i=1}^n X_i \right)
                            &= \frac{1}{n} \cdot
                                \e \left( \sum_{i=1}^n X_i \right) \\
                            &= \frac{1}{n} \cdot \sum_{i=1}^n \e(X_i) \\
                            &= \frac{1}{n} \cdot n \cdot \frac{7}{2}
                            = \frac{7}{2}
                        \end{split}
                    \end{equation*}
                    Für die Berechnung der Varianz nutzen wir die Unabhängigkeit
                    der einzelnen Würfe sowie $\var(aX) = a^2 \cdot \var(X)$.
                    \begin{equation*}
                        \begin{split}
                            \var \left(\frac{1}{n} \cdot \sum_{i=1}^n X_i\right)
                            &= \frac{1}{n^2} \cdot
                                \var \left(\sum_{i=1}^n X_i\right) \\
                            &= \frac{1}{n^2} \cdot \sum_{i=1}^n \var(X_i) \\
                            &= \frac{1}{n^2} \cdot n \cdot \frac{35}{12}
                            = \frac{35}{12 \cdot n}
                        \end{split}
                    \end{equation*}
                \end{proof}

            \item
                Sei $\overline{X} = \frac{1}{n} \sum_{i=1}^n X_i$.
                \begin{behaupt}
                    Ab $n = 1459$ Würfen ist die Wahrscheinlichkeit, dass das
                    arithmetische Mittel mehr als $\num{0,2}$ von seinem
                    Erwartungswert abweicht, kleiner als \num{0,05}:
                \begin{equation}
                    P\left\lbrace|\overline{X} - \e(\overline{X})| \geq \num{0,2}\right\rbrace
                    \leq \num{0,05}
                    \label{eq:foo}
                \end{equation}
                \end{behaupt}
                \begin{proof}
                Zu wählen ist ein möglichst kleines $n \in \mathbb{N}$, so dass
                Gleichung \eqref{eq:foo} gilt.
                Wir schätzen die Wahrscheinlichkeit mit der
                Chebyshev-Ungleichung nach oben hin ab
                \begin{equation*}
                    P\left\lbrace|\overline{X} - \e(\overline{X})| \geq \num{0,2}\right\rbrace
                    \leq \frac{\var(\overline{X})}{\num{0,2}^2}
                \end{equation*}
                und wählen $n$ so, dass
                \begin{equation*}
                    \frac{\var(\overline{X})}{\num{0,2}^2}
                    \leq \num{0,05}
                \end{equation*}
                gilt und damit insbesondere auch die Gleichung \eqref{eq:foo}.
                \begin{equation*}
                    \begin{alignedat}{2}
                        && \frac{\var(\overline{X})}{\num{0,2}^2}
                        &\leq \num{0,05} \\
                        &\gdw\ & \frac{35}{\num{0,2}^2 \cdot 12 \cdot n}
                        &\leq \num{0,05} \\
                        &\gdw\ &\frac{35}{\num{0,2}^2 \cdot 12 \cdot \num{0,05}}
                        &\leq n \\
                        &\gdw\ & \num{1458,33}
                        &\lessapprox n \\
                    \end{alignedat}
                \end{equation*}
                Wir wählen daher $n = 1459$.
                \end{proof}

            \item Bonusaufgabe \\
                Das folgende Skript simuliert 1000 Runden mit je 500 Würfen und
                zählt die Anzahl der Runden, in denen das arithmetische Mittel
                um mindestens \num{0,2} von dem oben berechneten Erwartungswert
                abweicht.
                Der Anteil dieser Runden wird auf der Standardausgabe
                ausgegeben.
                \begin{verbatim}
                    $ ./bonusaufgabe.py
                    0.012
                \end{verbatim}
                Mit dem verwendeten Seed für den Pseudozufallszahlengenerator
                gab es bei 12 von 1000 Runden eine Abweichung des arithmetischen
                Mittels von mehr als \num{0,2} vom Erwartungswert.

                \lstinputlisting[
                    language=Python,
                    frame=single,
                    numbers=left,
                    caption=bonusaufgabe.py
                ]
                {bonusaufgabe.py}

        \end{enumerate}

    \item
        \begin{behaupt}
            Das schwache Gesetz der großen Zahlen gilt im Allgemeinen nicht für
            nicht unkorrelierte Zufallsvariablen. \\
            Es existieren also nicht unkorrelierte Zufallsvariablen
            $X_1, X_2, \dotsc$ für die nicht für alle $\epsilon > 0$
            \begin{equation}
                P\left\{
                    \frac{1}{n} \left| \sum_{i=0}^n (X_i - \e(X_i)) \right|
                    \geq \epsilon
                \right\}
                \to 0
            \end{equation}
            für $n \to \infty$ gilt.
        \end{behaupt}
        \begin{proof}
            Wir wählen die Zufallsvariablen $X_1 = X_2 = \dotsb = X_n$ auf einem
            diskreten Wahrscheinlichkeitsraum.
            Für alle $1 \leq i, j \leq n$ gelte $P\{X_i = 0\} = P\{X_i = 1\} =
            \num{0,5}$ und damit $\e(X_i) = \num{0,5}$ und $\var(X_i) =
            \cov(X_i, Y_j) = \num{0,25}$.
            Die Zufallsvariablen sind also nicht unkorreliert und nehmen
            entweder alle den Wert 1 oder alle den Wert 0 an.
            Der Term $X_i - \e(X_i)$ ist im ersten Fall für alle
            $1 \leq i \leq n$ gleich \num{0,5} und im zweiten \num{-0,5}.
            \begin{equation}
                \begin{split}
                    P\left\{
                        \frac{1}{n} \left| \sum_{i=1}^n (X_i - \e(X_i)) \right|
                        \geq \epsilon
                    \right\}
                    &=
                    P\left\{
                        \frac{1}{n} \cdot n \cdot \num{0,5} \geq \epsilon
                    \right\} \\
                    &= P\left\{ \num{0,5} \geq \epsilon \right\} \\
                    &=
                    \begin{cases}
                        1 & \text{, wenn } \epsilon \leq \num{0,5} \\
                        0 & \text{, wenn } \epsilon > \num{0,5} \\
                    \end{cases}
                \end{split}
                \label{eq:bar}
            \end{equation}
            Es gibt also $\epsilon > 0$, für die Gleichung \eqref{eq:bar} nicht
            gegen 0 strebt, wenn $n \to \infty$:
            alle $0 < \epsilon \leq \num{0,5}$.
        \end{proof}

\end{enumerate}


\end{document}
