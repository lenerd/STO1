\documentclass[a4paper]{scrartcl}

% font/encoding packages
\usepackage[utf8]{inputenc}
\usepackage[T1]{fontenc}
\usepackage{lmodern}
\usepackage[ngerman]{babel}
\usepackage[ngerman=ngerman-x-latest]{hyphsubst}

\usepackage{amsmath, amssymb, amsfonts, amsthm}
\usepackage{array}
\usepackage{stmaryrd}
\usepackage{marvosym}
\allowdisplaybreaks
\usepackage[output-decimal-marker={,}]{siunitx}
\usepackage[shortlabels]{enumitem}
\usepackage[section]{placeins}
\usepackage{float}
\usepackage{units}

\newtheorem*{behaupt}{Behauptung}
\newcommand{\gdw}{\Leftrightarrow}
\newcommand{\N}{\mathbb{N}}
\newcommand{\cov}{\operatorname{Cov}}
\newcommand{\e}{\operatorname{E}}
\newcommand{\var}{\operatorname{Var}}
\newcommand{\corr}{\operatorname{Corr}}

\usepackage{fancyhdr}
\pagestyle{fancy}
\lhead{Stochastik 1 - Gruppe 5 - Blatt 9}
\rhead{Lennart Braun, Merlin Koglin, Kai Robin Sachse}
\cfoot{\thepage}


\title{Stochastik 1 für Informatiker}
\subtitle{Blatt 9 Hausaufgaben (Gruppe 5)}
\author{
    Lennart Braun (6523742), \\
    Merlin Koglin (6450362), \\
    Kai Robin Sachse (6450486)
}
\date{zum 23. Juni 2015}

\begin{document}
\maketitle

\begin{enumerate}[label=\bfseries\arabic*.]
    \item
        Sei $X$ die Anzahl an Zugriffen pro Zeiteinheit auf einen Server.
        Es gelten $\e(X) = 400$ und $\var(X) = 400$.
        \begin{behaupt}
            Mit der Markov-Ungleichung lässt die die Wahrscheinlichkeit für
            mindestens 1200 Zugriffe in der nächsten Zeiteinheit abschätzen als:
            \begin{equation*}
                P\{X \geq 1200\} \leq \frac{1}{3}
            \end{equation*}
        \end{behaupt}
        \begin{proof}
            Da $X \geq 0$ gilt $X = |X|$.
            In die Markov-Ungleichung eingesetzt, ergibt sich folgendes.
            \begin{equation*}
                P\{X \geq 1200\} = P\{|X| \geq 1200\}
                \leq \frac{\e(X)}{1200} = \frac{400}{1200} = \frac{1}{3}
            \end{equation*}
        \end{proof}

        \begin{behaupt}
            Mit der Chebyshev-Ungleichung lässt die die Wahrscheinlichkeit für
            mindestens 1200 Zugriffe in der nächsten Zeiteinheit abschätzen als:
            \begin{equation*}
                P\{X \geq 1200\} \leq \frac{1}{1600}
            \end{equation*}
        \end{behaupt}
        \begin{proof}
            \begin{equation*}
                \begin{alignedat}{2}
                    P\{X \geq 1200\} &= P\{X - \e(X) \geq 800\} \\
                                     &\leq P\{|X - \e(X)| \geq 800\}
                    \qquad &\text{da } X - \e(X) \leq |X - \e(X)| \\
                    &\leq \frac{\var(X)}{800^2}
                    \qquad &\text{Chebyshev's Ungleichung} \\
                    &= \frac{400}{800^2} = \frac{1}{1600}
                \end{alignedat}
            \end{equation*}
        \end{proof}

    \item

    \item

\end{enumerate}


\end{document}
