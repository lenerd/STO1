\documentclass[a4paper]{scrartcl}

% font/encoding packages
\usepackage[utf8]{inputenc}
\usepackage[T1]{fontenc}
\usepackage{lmodern}
\usepackage[ngerman]{babel}
\usepackage[ngerman=ngerman-x-latest]{hyphsubst}

\usepackage{amsmath, amssymb, amsfonts, amsthm}
\usepackage{stmaryrd}
\usepackage{marvosym}
\allowdisplaybreaks
\usepackage[output-decimal-marker={,}]{siunitx}
\usepackage[shortlabels]{enumitem}
\usepackage[section]{placeins}
\usepackage{float}

\newtheorem*{behaupt}{Behauptung}
\newcommand{\gdw}{\Leftrightarrow}

\usepackage{fancyhdr}
\pagestyle{fancy}
\lhead{Stochastik 1 - Gruppe 5 - Blatt 5}
\rhead{Lennart Braun, Merlin Koglin, Kai Robin Sachse}
\cfoot{\thepage}


\title{Stochastik 1 für Informatiker}
\subtitle{Blatt 5 Hausaufgaben (Gruppe 5)}
\author{
    Lennart Braun (6523742), \\
    Merlin Koglin (6450362), \\
    Kai Robin Sachse (6450486)
}
\date{zum 12. Mai 2015}

\begin{document}
\maketitle

\begin{enumerate}[label=\bfseries\arabic*.]
    \item
        Gegeben sei ein Test mit 12 Fragen und je vier Antwortmöglichkeiten von denen
        je genau eine richtig ist. Der Test wird bestanden (Ereignis A) bei mindestens
        7 richtigen Antworten.
        \begin{enumerate}[label=(\alph*)]
            \item
                \begin{behaupt}
                    Ein ahnungsloser Student besteht den Test mit W. BLUBB
                \end{behaupt}
                \begin{proof}
                    Wir betrachten den Grundraum $\Omega=\{0,1\}^12$ mit
                    $\omega=(\omega_1,\ldots,\omega_12)$ wobei $\omega_i=1$ bedeutet die $i$-te
                    Antwort war richtig und 0 wenn falsch. Die Wahrscheinlichkeit die richtige
                    Antwort zu wählen liegt jeweils bei $p=0,25$. Da die Auswahlen der Antwort
                    unabhängig voneinander sind und wir das selbe Ereignis 12mal durchführen,
                    ist das GesamtWmaß die Binomialverteilung womit die Zähldichte gegeben ist durch
                    \[f(k)=\binom{12}{k} \frac{1}{4^k}\left(\frac{3}{4}\right)^{12-k} = 
                    \binom{12}{k} \frac{3^{12-k}}{4^{12}}= \frac{1}{4^{12}}\binom{12}{k} 3^{12-k}\]
                    wobei $k$ die Anzahl der richtigen Antworten sei. Die gesuchte W. ergibt sich als
                    Summe über die Anzahl der richtigen Antworten:
                    \[P(A)=\sum_{k=7}^{12}f(k)=\sum_{k=7}^{12} \frac{1}{4^{12}}\binom{12}{k} 3^{12-k}=
                     \frac{1}{4^{12}}\sum_{k=7}^{12}\binom{12}{k} 3^{12-k}\approx 0,0143\]
                    
                \end{proof}

            \item
                \begin{behaupt}
                    Ein Student der je zwei falsche Antworten erkennt besteht mit W. BLUBB
                \end{behaupt}
                \begin{proof}
                    Der Beweis verläuft analog zu $a)$ nur das nun $p=0,5$ gilt. Somit haben wir
                    \[P(A)=\sum_{k=7}^{12}f(k)=\sum_{k=7}^{12} \frac{1}{2^{12}}\binom{12}{k} 1^{12-k}=
                     \frac{1}{2^{12}}\sum_{k=7}^{12}\binom{12}{k} \approx 0,3872\]
                \end{proof}

        \end{enumerate}

    \item	        
        Seien $X:\Omega\rightarrow S_1,Y:\Omega\rightarrow S_2$ Zv. auf disk. Wraum $(\Omega,P)$
        mit $S_1,S_2$ höchstens abz. und $f_{(X,Y)}(x,y)=P\{X=x,Y=y\}\; x\in S_1,y\in S_2$
        Zähldichte von $(X,Y)$
        \begin{enumerate}[label=(\alph*)]
            \item
                \begin{behaupt}
                    Es gilt
                    \begin{align*}
                    f_X(x)=\sum_{y\in S_2} f_{(X,Y)}(x,y),\; x\in S_1\\
                    f_Y(y)=\sum_{x\in S_1} f_{(X,Y)}(x,y),\; y\in S_2\\
                    \end{align*}
                \end{behaupt}
                \begin{proof}
                
                \end{proof}
                
            \item
                \begin{behaupt}
                    Für $S_1=\{0,1,2\}$ und $S_2=\{0,1\}$ gilt $f_X(x)=$ BLUBB und $f_Y(y)=$ BLUBB
                \end{behaupt}
                \begin{proof}
                    
                \end{proof}
                
            \item
                \begin{behaupt}
                    $X,Y$ sind (nicht) stoch unabh.
                \end{behaupt}
                \begin{proof}
                    
                \end{proof}
        \end{enumerate}

                

\end{enumerate}

\end{document}
