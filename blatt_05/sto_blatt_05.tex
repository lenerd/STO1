\documentclass[a4paper]{scrartcl}

% font/encoding packages
\usepackage[utf8]{inputenc}
\usepackage[T1]{fontenc}
\usepackage{lmodern}
\usepackage[ngerman]{babel}
\usepackage[ngerman=ngerman-x-latest]{hyphsubst}

\usepackage{amsmath, amssymb, amsfonts, amsthm}
\usepackage{stmaryrd}
\usepackage{marvosym}
\allowdisplaybreaks
\usepackage[output-decimal-marker={,}]{siunitx}
\usepackage[shortlabels]{enumitem}
\usepackage[section]{placeins}
\usepackage{float}
\usepackage{units}

\newtheorem*{behaupt}{Behauptung}
\newcommand{\gdw}{\Leftrightarrow}

\usepackage{fancyhdr}
\pagestyle{fancy}
\lhead{Stochastik 1 - Gruppe 5 - Blatt 5}
\rhead{Lennart Braun, Merlin Koglin, Kai Robin Sachse}
\cfoot{\thepage}


\title{Stochastik 1 für Informatiker}
\subtitle{Blatt 5 Hausaufgaben (Gruppe 5)}
\author{
    Lennart Braun (6523742), \\
    Merlin Koglin (6450362), \\
    Kai Robin Sachse (6450486)
}
\date{zum 12. Mai 2015}

\begin{document}
\maketitle

\begin{enumerate}[label=\bfseries\arabic*.]
    \item
        Gegeben sei ein Test mit zwölf Fragen und je vier Antwortmöglichkeiten
        von denen je genau eine richtig ist.
        Der Test wird bestanden (Ereignis $A$) bei mindestens sieben richtigen
        Antworten.
        \begin{enumerate}[label=(\alph*)]
            \item
                \begin{behaupt}
                    Ein ahnungsloser Student besteht den Test mit einer
                    Wahrscheinlichkeit von ca. \num{0.0143}.
                \end{behaupt}
                \begin{proof}
                    Wir betrachten den Grundraum $\Omega = \{ 0, 1 \}^{12}$ mit
                    $\omega = (\omega_1, \ldots, \omega_{12})$ wobei
                    $\omega_i = 1$ bedeutet, dass die $i$-te Antwort richtig
                    war, und $\omega_i = 0$, dass sie falsch war.
                    Die Wahrscheinlichkeit, die richtige Antwort zu wählen,
                    liegt jeweils bei $p = \frac{1}{4}$.
                    Da die Auswahlen der Antwort unabhängig voneinander sind und
                    wir das selbe Zufallsexperiment $n = 12$ mal durchführen,
                    ist das Gesamtwahrscheinlichkeitsmaß die Binomialverteilung
                    $\mathcal{B}_{(12, \nicefrac{1}{4})}$.
                    Für die Zähldichte gilt bei der Binomialverteilung
                    $\mathcal{B}_{(n, p)}$ für $k$ erfolgreiche Antworten
                    \begin{equation*}
                        f_{\mathcal{B}_{(n, p)}}(k)
                        = \binom{n}{k} \cdot p^k \cdot (1 - p)^{n-k}
                        \text{ .}
                    \end{equation*}
                    Da der Test bei mindestens sieben richtigen Antworten als
                    bestanden gilt, ergibt sich die gesuchte Wahrscheinlichkeit
                    als Summe über alle Zähldichten von $7 \leq k \leq 12$.
                    Für unser Experiment gilt also.
                    \begin{equation*}
                        \begin{split}
                        P(A) &=
                        \sum_{k = 7}^{12} f_{\mathcal{B}_{(12, \frac{1}{4})}}(k)
                        = \sum_{k = 7}^{12} \binom{12}{k}
                            \cdot \left( \frac{1}{4} \right)^k
                            \cdot \left( \frac{3}{4} \right)^{12-k} \\
                        &= \frac{1}{4^{12}} \cdot
                            \sum_{k = 7}^{12} \binom{12}{k} \cdot 3^{12-k}
                            \approx \num{0.0143}
                        \end{split}
                    \end{equation*}
                \end{proof}

            \item
                \begin{behaupt}
                    Ein Student der je zwei falsche Antworten erkennt besteht
                    mit einer Wahrscheinlichkeit von ca. \num{0.3872}.
                \end{behaupt}
                \begin{proof}
                    Der Beweis verläuft analog zu $a)$ nur das nun
                    $p = \frac{1}{2}$ gilt.
                    Somit haben wir
                    \begin{equation*}
                        \begin{split}
                        P(A) &=
                        \sum_{k = 7}^{12} f_{\mathcal{B}_{(12, \frac{1}{2})}}(k)
                        = \sum_{k = 7}^{12} \binom{12}{k}
                            \cdot \left( \frac{1}{2} \right)^k
                            \cdot \left( \frac{1}{2} \right)^{12-k} \\
                        &= \frac{1}{2^{12}} \cdot \sum_{k = 7}^{12} \binom{12}{k}
                        \approx \num{0,3872}
                        \end{split}
                    \end{equation*}
                \end{proof}

        \end{enumerate}

    \item
        Seien $X \colon \Omega \to S_1, Y \colon \Omega \to S_2$
        Zufallsvariablen auf einem diskreten Wahrscheinlichkeitsraum
        $(\Omega, P)$ un seien $S_1, S_2$ höchstens abzählbar.

        Sei $f_{(X,Y)}(x,y) = P\{X = x, Y = y\}\; x \in S_1, y \in S_2$ die
        Zähldichte von $(X, Y)$.
        \begin{enumerate}[label=(\alph*)]
            \item
                \begin{behaupt}
                    Seien $f_X(x) = P\{X = x\}, x \in S_1$ und
                    $f_Y(y) = P\{Y = Y\}, y \in S_2$ die Zähldichen von $X$ bzw.
                    $Y$.
                    Es gelten folgende Gleichungen:
                    \begin{align*}
                    f_X(x) = \sum_{y \in S_2} f_{(X,Y)}(x, y), \; x \in S_1 \\
                    f_Y(y) = \sum_{x \in S_1} f_{(X,Y)}(x, y), \; y \in S_2 \\
                    \end{align*}
                \end{behaupt}
                \begin{proof}
                    Es sei $x \in S_1$.
                    \begin{equation*}
                        \begin{split}
                            f_X(x) &= P^X\{ x \} \\
                            &= P\{ X = x \} \\
                            &= P\{ \omega \in \Omega \ |\ X(\omega) = x \} \\
                            &\stackrel{(\star)}{=} P \left( \bigcup_{y \in S_2}
                                \{ \omega \in \Omega \ |\ X(\omega) = x
                                \land Y(\omega) = y \} \right) \\
                            &= \sum_{y \in S_2} P \{ \omega \in \Omega \ |\ 
                                X(\omega) = x \land Y(\omega) = y \} \\
                            &= \sum_{y \in S_2} P \{ X = x \land Y = y \} \\
                            &= \sum_{y \in S_2} P \{ (X, Y) = (x, y) \} \\
                            &= \sum_{y \in S_2} P^{(X, Y)} \{ (x, y) \} \\
                            &= \sum_{y \in S_2} f_{(X, Y)} (x, y)
                        \end{split}
                    \end{equation*}
                    $(\star)$ Die Menge wird durch $Y$ in disjunkte Teilmengen
                    partitioniert.

                    Für die zweite Gleichung wird analog für $y \in S_2$
                    vorgegangen:
                    \begin{equation*}
                        \begin{split}
                            f_Y(y) &= P^Y\{ y \} \\
                            &= P\{ Y = y \} \\
                            &= P\{ \omega \in \Omega \ |\ Y(\omega) = y \} \\
                            &= P \left( \bigcup_{x \in S_1}
                                \{ \omega \in \Omega \ |\ Y(\omega) = y
                                \land X(\omega) = x \} \right) \\
                            &= \sum_{x \in S_1} P \{ \omega \in \Omega \ |\ 
                                Y(\omega) = y \land X(\omega) = x \} \\
                            &= \sum_{x \in S_1} P \{ X = x \land Y = y \} \\
                            &= \sum_{x \in S_1} P \{ (X, Y) = (x, y) \} \\
                            &= \sum_{x \in S_1} P^{(X, Y)} \{ (x, y) \} \\
                            &= \sum_{x \in S_1} f_{(X, Y)} (x, y)
                        \end{split}
                    \end{equation*}

                \end{proof}

            \item Sei nun $S_1 = \{ 0, 1, 2 \}$ und $S_2 = \{ 0, 1 \}$.
                Sei die Zähldichte $f_{(X, Y)}(x, y)$ von $(X, Y)$ in Tabelle
                \ref{tab:zdichte-fxy} angegeben.
                \begin{table}[H]
                    \centering
                    \begin{tabular}{c|ccc}
                            &     & $x$ &     \\
                        $y$ & $0$ & $1$ & $2$ \\ \hline
                        $0$ & $\nicefrac{1}{4}$ & $\nicefrac{1}{6}$
                        & $\nicefrac{1}{12}$ \\
                        $1$ & $\nicefrac{1}{6}$ & $\nicefrac{1}{9}$
                        & $\nicefrac{2}{9}$ \\
                    \end{tabular}
                    \caption{Wertetabelle der Zähldichte $f_{(X, Y)}$}
                    \label{tab:zdichte-fxy}
                \end{table}
                \begin{enumerate}[label=(\alph{enumi}.\arabic*)]
                    \item
                        \begin{behaupt}
                            Für $f_X$ und $f_Y$ gelten die folgenden
                            Wertetabellen.
                            \begin{table}[H]
                                \centering
                                \begin{tabular}{c|ccc}
                                    $x$ & $0$ & $1$ & $2$ \\ \hline
                                    $f_X(x)$ & $\nicefrac{15}{36}$
                                        & $\nicefrac{10}{36}$
                                        & $\nicefrac{11}{36}$ \\
                                \end{tabular}
                                \quad
                                \begin{tabular}{c|cc}
                                    $y$ & $0$ & $1$ \\ \hline
                                    $f_Y(y)$ & $\nicefrac{1}{2}$
                                        &  $\nicefrac{1}{2}$ \\
                                \end{tabular}
                                \caption{Wertetabellen der Zähldichten $f_X$ und
                                    $f_Y$}
                                \label{tab:zdichten-fx-fy}
                            \end{table}
                        \end{behaupt}
                        \begin{proof}
                            Wir verwenden die Gleichungen aus der vorherigen
                            Aufgabe.
                            \begin{align*}
                                \begin{split}
                                    f_X(0) &= \sum_{y \in S_2} f_{(X, Y)}(0, y)
                                    = f_X(0, 0) + f_X(0, 1) \\
                                    &= \frac{1}{4} + \frac{1}{6}
                                    = \frac{5}{12} = \frac{15}{36}
                                \end{split} \\
                                \begin{split}
                                    f_X(1) &= \sum_{y \in S_2} f_{(X, Y)}(1, y)
                                    = f_X(1, 0) + f_X(1, 1) \\
                                    &= \frac{1}{6} + \frac{1}{9}
                                    = \frac{5}{18} = \frac{10}{36}
                                \end{split} \\
                                \begin{split}
                                    f_X(2) &= \sum_{y \in S_2} f_{(X, Y)}(2, y)
                                    = f_X(2, 0) + f_X(2, 1) \\
                                    &= \frac{1}{12} + \frac{2}{9}
                                    = \frac{11}{36}
                                \end{split} \\
                                \begin{split}
                                    f_Y(0) &= \sum_{x \in S_1} f_{(X, Y)}(x, 0)
                                    = f_Y(0, 0) + f_Y(1, 0) + f_Y(2, 0) \\
                                    &= \frac{1}{4} + \frac{1}{6} + \frac{1}{12}
                                    = \frac{6}{12} = \frac{1}{2}
                                \end{split} \\
                                \begin{split}
                                    f_Y(1) &= \sum_{x \in S_1} f_{(X, Y)}(x, 1)
                                    = f_Y(0, 1) + f_Y(1, 1) + f_Y(2, 1) \\
                                    &= \frac{1}{6} + \frac{1}{9} + \frac{2}{9}
                                    = \frac{9}{18} = \frac{1}{2}
                                \end{split}
                            \end{align*}
                        \end{proof}

                    \item
                        \begin{behaupt}
                            $X$ und $Y$ sind nicht stochastisch unabhängig.
                        \end{behaupt}
                        \begin{proof}
                            Angenommen $X$ und $Y$ wären stochastisch
                            unabhängig.
                            Dann müsste nach Satz 3.4 für die Zähldichte
                            $f_{(X, Y)}$ von $P^{(X, Y)}$ und die Zähldichten
                            $f_X$ und $f_Y$ von $P^Y$ bzw. $P^Y$ gelten:
                            \begin{equation*}
                                f_{(X, Y)} (x, y) = f_X(x) \cdot f_Y(y)
                                \text{ für alle } x \in S_1, y \in S_2
                            \end{equation*}
                            Da die Gleichung für alle $x, y$ gelten muss, so
                            muss sie insbesondere auch für $x = y = 0$ gelten.
                            \begin{equation*}
                                f_{(X, Y)}(0, 0) = \frac{1}{4} = \frac{6}{24}
                                \neq
                                \frac{5}{24} = \frac{15}{36} \cdot \frac{1}{2}
                                = f_X(0) \cdot f_Y(0)
                            \end{equation*}
                            Da schon diese Belegung von $x$ und $y$ die
                            Gleichung nicht erfüllt, gilt diese erst recht nicht
                            für alle $x, y$.
                            Dies steht im Widerspruch zu obiger Annahme, $X$ und
                            $Y$ wären stochastisch unabhängig.
                            Daher sind $X$ und $Y$ nicht stochastisch
                            unabhängig.
                        \end{proof}
                \end{enumerate}
        \end{enumerate}

\end{enumerate}

\end{document}
