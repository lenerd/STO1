\documentclass[a4paper]{scrartcl}

% font/encoding packages
\usepackage[utf8]{inputenc}
\usepackage[T1]{fontenc}
\usepackage{lmodern}
\usepackage[ngerman]{babel}
\usepackage[ngerman=ngerman-x-latest]{hyphsubst}

\usepackage{amsmath, amssymb, amsfonts, amsthm}
\usepackage{stmaryrd}
\usepackage{marvosym}
\allowdisplaybreaks
\usepackage[output-decimal-marker={,}]{siunitx}
\usepackage[shortlabels]{enumitem}
\usepackage[section]{placeins}
\usepackage{float}
\usepackage{units}

\newtheorem*{behaupt}{Behauptung}
\newcommand{\gdw}{\Leftrightarrow}
\newcommand{\N}{\mathbb{N}}

\usepackage{fancyhdr}
\pagestyle{fancy}
\lhead{Stochastik 1 - Gruppe 5 - Bonusblatt}
\rhead{Lennart Braun, Merlin Koglin, Kai Robin Sachse}
\cfoot{\thepage}


\title{Stochastik 1 für Informatiker}
\subtitle{Bonusblatt (Gruppe 5)}
\author{
    Lennart Braun (6523742), \\
    Merlin Koglin (6450362), \\
    Kai Robin Sachse (6450486)
}
\date{zum 2. Juni 2015}

\begin{document}
\maketitle

\begin{enumerate}[label=\bfseries\arabic*.]
    \item
        Seien $X$ und $Y$ zwei stochastisch unabhängige Zufallsvariablen auf
        einem diskreten Wahrscheinlichkeitsraum $(\Omega, P)$.
        Es gelten $X \sim \mathcal{P}_{\lambda_1}$ und
        $Y \sim \mathcal{P}_{\lambda_2}$ mit $\lambda_1, \lambda_2 > 0$.
        \begin{behaupt}
            Für alle $s \in \mathbb{N}_0$ und alle $k \in \{0, \ldots, s\}$ gilt
            \begin{equation*}
                P(X = k \ |\ X + Y = s) = \binom{s}{k} \cdot
                \left( \frac{\lambda_1}{\lambda_1 + \lambda_2} \right)^k \cdot
                \left( 1 - \frac{\lambda_1}{\lambda_1 + \lambda_2} \right)^{s-k}
            \end{equation*}
        \end{behaupt}
        \begin{proof}
            \begin{equation*}
                \begin{split}
                    P(X = k \ |\ X + Y = s)
                    &= \frac{P(\{X = k\} \cap \{X + Y = s\})}{P(\{X + Y = s\})}
                    \\
                    &= \frac{P\{X = k \land Y = s - k\}}{P\{X + Y = s\}} \\
                    &= \frac{P\{X = k\} \cdot P\{Y = s - k\}}{P\{X + Y = s\}}
                    \qquad \text{(da $X$ und $Y$ st. unabh.)} \\
                    &= \frac{f_{\lambda_1}(k) \cdot f_{\lambda_2}(s - k)}
                            {f_{\lambda_1 + \lambda_2}(s)} \\
                    &= \frac{e^{-\lambda_1} \cdot \frac{\lambda_1^k}{k!} \cdot
                        e^{-\lambda_2} \cdot \frac{\lambda_2^{s-k}}{(s-k)!}}
                        {e^{-(\lambda_1 + \lambda_2)} \cdot
                        \frac{(\lambda_1 + \lambda_2)^{s}}{s!}} \\
                    &= \frac{s!}{k! \cdot (s-k)!} \cdot
                        \frac{\lambda_1^k \cdot \lambda_2^{s-k}}
                        {(\lambda_1 + \lambda_2)^s} \\
                    &= \binom{s}{k} \cdot
                        \frac{\lambda_1^k}{(\lambda_1 + \lambda_2)^k} \cdot
                        \frac{\lambda_2^{s-k}}{(\lambda_1 + \lambda_2)^{s-k}} \\
                    &= \binom{s}{k} \cdot
                        \left( \frac{\lambda_1}{\lambda_1 + \lambda_2} \right)^k
                        \cdot
                        \left(\frac{\lambda_2}{\lambda_1+\lambda_2}\right)^{s-k}
                        \\
                    &= \binom{s}{k} \cdot
                        \left( \frac{\lambda_1}{\lambda_1 + \lambda_2} \right)^k
                        \cdot \left(1 - \frac{\lambda_1}{\lambda_1 + \lambda_2}
                        \right)^{s-k}
                \end{split}
            \end{equation*}
        \end{proof}

    \item
        Es sei $X\colon \Omega \to \mathbb{N}_0$ eine Zufallsvariable auf einem
        diskreten Wahrscheinlichkeitsraum $(\Omega, P)$.
        Der Erwartungswert von $X$ existiere.
        \begin{enumerate}[label=(\alph*)]
            \item
                \begin{behaupt}
                    \begin{equation*}
                        E(X) = \sum_{k=1}^\infty P\{X \geq k\}
                    \end{equation*}
                \end{behaupt}
                \begin{proof}
                    \begin{equation*}
                        \begin{split}
                            E(X)
                            &= \sum_{x \in X(\Omega)} x \cdot P\{X = x\} \\
                            &= \sum_{l = 0}^\infty l \cdot P\{X = l\} \\
                            &= \sum_{l = 1}^\infty l \cdot P\{X = l\} \\
                            &= \sum_{l = 1}^\infty \sum_{k = 1}^l P\{X = l\} \\
                            &= \sum_{k=1}^\infty \sum_{l=k}^\infty P\{X = l\} \\
                            &= \sum_{k=1}^\infty P\{X \geq k\} \\
                        \end{split}
                    \end{equation*}
                \end{proof}

            \item
                Es gelten nun $P\{X = l\} = f_X(l) = (1 - p)^l \cdot p$ für alle
                $l \in \mathbb{N}_0$ und ein $p \in (0, 1]$.
                \begin{behaupt}
                    Es gilt
                    \begin{equation*}
                        E(X) = \frac{1 - p}{p}
                        \text{ .}
                    \end{equation*}
                \end{behaupt}
                \begin{proof}
                    Zunächst wird gezeigt, dass $P\{X \geq k\} = (1 - p)^k$
                    gilt.
                    \begin{equation*}
                        \begin{split}
                            P\{X \geq k\}
                            &= \sum_{l=k}^\infty (1 - p)^l \cdot p \\
                            &= p \cdot \sum_{l=k}^\infty (1 - p)^l \\
                            &= p \cdot \left( \sum_{l=0}^\infty (1 - p)^l
                                - \sum_{l=0}^{k-1} (1 - p)^l \right) \\
                            &= p \cdot \left( \frac{1}{1 - (1 -p)}
                                - \frac{1 - (1 - p)^k}{1 - (1 - p)} \right) \\
                            &= p \cdot \left( \frac{1 - 1 + (1 - p)^k}
                                {p} \right) \\
                            &= (1 - p)^k
                        \end{split}
                    \end{equation*}
                    Mit dieser Aussage und der in Aufgabenteil (a) bewiesenen
                    lässt sich der Erwartungswert $E(X)$ berechnen.
                    \begin{equation*}
                        \begin{split}
                            E(X)
                            &= \sum_{k=1}^\infty P\{X \geq k\} \\
                            &= \sum_{k=1}^\infty (1 - p)^k \\
                            &= \sum_{k=0}^\infty \left( (1 - p)^k \right)
                                - (1 - p)^0 \\
                            &= \frac{1}{1 - (1 - p)} - 1 \\
                            &= \frac{1}{p} - 1 \\
                            &= \frac{1 - p}{p}
                        \end{split}
                    \end{equation*}
                \end{proof}

        \end{enumerate}
\end{enumerate}


\end{document}
