\documentclass[a4paper]{scrartcl}

% font/encoding packages
\usepackage[utf8]{inputenc}
\usepackage[T1]{fontenc}
\usepackage{lmodern}
\usepackage[ngerman]{babel}
\usepackage[ngerman=ngerman-x-latest]{hyphsubst}

\usepackage{amsmath, amssymb, amsfonts, amsthm}
\usepackage{stmaryrd}
\usepackage{marvosym}
\allowdisplaybreaks
\usepackage[output-decimal-marker={,}]{siunitx}
\usepackage[shortlabels]{enumitem}
\usepackage[section]{placeins}
\usepackage{float}
\usepackage{units}

\newtheorem*{behaupt}{Behauptung}
\newcommand{\gdw}{\Leftrightarrow}
\newcommand{\N}{\mathbb{N}}

\usepackage{fancyhdr}
\pagestyle{fancy}
\lhead{Stochastik 1 - Gruppe 5 - Bonusblatt}
\rhead{Lennart Braun, Merlin Koglin, Kai Robin Sachse}
\cfoot{\thepage}


\title{Stochastik 1 für Informatiker}
\subtitle{Bonusblatt (Gruppe 5)}
\author{
    Lennart Braun (6523742), \\
    Merlin Koglin (6450362), \\
    Kai Robin Sachse (6450486)
}
\date{zum 2. Juni 2015}

\begin{document}
\maketitle

\begin{enumerate}[label=\bfseries\arabic*.]
    \item
        Seien $X$ und $Y$ zwei stochastisch unabhängige Zufallsvariablen auf
        einem diskreten Wahrscheinlichkeitsraum $(\Omega, P)$.
        Es gelten $X \sim \mathcal{P}_{\lambda_1}$ und
        $Y \sim \mathcal{P}_{\lambda_2}$ mit $\lambda_1, \lambda_2 > 0$.
        \begin{behaupt}
            Für alle $s \in \mathbb{N}_0$ und alle $k \in \{0, \ldots, s\}$ gilt
            \begin{equation*}
                P \left( X = k \ |\ X + Y = s \right) = \binom{s}{k} \cdot
                \left( \frac{\lambda_1}{\lambda_1 + \lambda_2} \right)^k \cdot
                \left( 1 - \frac{\lambda_1}{\lambda_1 + \lambda_2} \right)^{s-k}
            \end{equation*}
        \end{behaupt}
        \begin{proof}
            TODO
        \end{proof}

    \item
        Es sei $X\colon \Omega \to \mathbb{N}_0$ eine Zufallsvariable auf einem
        diskreten Wahrscheinlichkeitsraum $(\Omega, P)$.
        Der Erwartungswert von $X$ existiere.
        \begin{enumerate}[label=(\alph*)]
            \item
                \begin{behaupt}
                    \begin{equation*}
                        E(X) = \sum_{k=1}^\infty P\{X \geq k\}
                    \end{equation*}
                \end{behaupt}
                \begin{proof}
                    TODO
                \end{proof}

            \item
                Es gelten nun $P\{X = l\} = f_X(l) = (1 - p)^i \cdot p$ für alle
                $l \in \mathbb{N}_0$ und ein $p \in (0, 1]$.
                \begin{behaupt}
                    Es gilt
                    \begin{equation*}
                        E(X) = TODO
                        \text{ .}
                    \end{equation*}
                \end{behaupt}

        \end{enumerate}
\end{enumerate}


\end{document}
