\documentclass[a4paper]{scrartcl}

% font/encoding packages
\usepackage[utf8]{inputenc}
\usepackage[T1]{fontenc}
\usepackage{lmodern}
\usepackage[ngerman]{babel}
\usepackage[ngerman=ngerman-x-latest]{hyphsubst}

\usepackage{amsmath, amssymb, amsfonts, amsthm}
\usepackage{array}
\usepackage{stmaryrd}
\usepackage{marvosym}
\allowdisplaybreaks
\usepackage[output-decimal-marker={,}]{siunitx}
\usepackage[shortlabels]{enumitem}
\usepackage[section]{placeins}
\usepackage{float}
\usepackage{units}
\usepackage{listings}
\usepackage{pgfplots}
\pgfplotsset{compat=1.12}

\newtheorem*{behaupt}{Behauptung}
\newcommand{\gdw}{\Leftrightarrow}
\newcommand{\N}{\mathbb{N}}
\newcommand{\cov}{\operatorname{Cov}}
\newcommand{\e}{\operatorname{E}}
\newcommand{\var}{\operatorname{Var}}
\newcommand{\corr}{\operatorname{Corr}}

\usepackage{fancyhdr}
\pagestyle{fancy}
\lhead{Stochastik 1 - Gruppe 5 - Blatt 11}
\rhead{Lennart Braun, Merlin Koglin, Kai Robin Sachse}
\cfoot{\thepage}


\title{Stochastik 1 für Informatiker}
\subtitle{Blatt 11 Hausaufgaben (Gruppe 5)}
\author{
    Lennart Braun (6523742), \\
    Merlin Koglin (6450362), \\
    Kai Robin Sachse (6450486)
}
\date{zum 7. Juni 2015}

\begin{document}
\maketitle

\begin{enumerate}[label=\bfseries\arabic*.]
    \item
        \begin{enumerate}[label=(\alph*)]
            \item
                Da das Gewicht nicht auf diskrete Werte abgebildet werden kann,
                würden wir den Grundraum
                $\Omega = \{ x \in \mathbb{R}\ |\ 0 < x \leq 10 \}$ wählen. 
                Wobei $\omega \in \Omega$ ein mögliches Gewicht in Kilogramm
                repräsentiert.
                $\Omega$ ist überabzählbar unendlich.

            \item
                Die Anzahl an Sechsten beim zehnmaligen Würfeln ist beschränkt
                und kann nur Zahlen $n \in \{0, \dotsc, 10\} =: \Omega$
                annehmen.
                Es handelt sich um einen endlichen Grundraum mit
                $|\Omega| = 11$.

            \item
                Für die Anzahl an Fehlversuchen, bis man das erste Mal eine
                Sechs beim „Mensch ärgere Dich nicht“ würfelt, wählen wir die
                natürlichen Zahlen $\mathbb{N} \cup \{0\}$ als Grundraum
                $\Omega$.
                Vor der ersten Sechs können beliebig viele Würfe andere
                Augenzahlen ergeben.
                $\Omega$ ist abzählbar unendlich.
                

        \end{enumerate}

    \item
        Gegeben sei die Funktion $F$:
        \begin{equation*}
            F(x) =
            \begin{cases}
                0, & x \leq 1 \\
                \num{0,5} \cdot x - \num{0,5}, & 1 < x \leq 3 \\
                1, & 3 < x \\
            \end{cases}
        \end{equation*}
        \begin{enumerate}[label=(\alph*)]
            \item \hfill \\
                \begin{figure}[H]
                    \centering
                    \begin{tikzpicture}
                        \begin{axis}[
                            axis lines=left,
                            xlabel=$x$,
                            ylabel=$F(x)$,
                        ]
                            \addplot[
                                domain=0:1,
                            ]
                            {0};
                            \addplot[
                                domain=1:3,
                            ]
                            {0.5*x-0.5};
                            \addplot[
                                domain=3:5,
                            ]
                            {1};
                        \end{axis}
                    \end{tikzpicture}
                    \caption{Plot von $F$}
                \end{figure}

            \item
                \begin{behaupt}
                    $F$ ist eine Verteilungsfunktion.
                \end{behaupt}
                \begin{proof}
                    Um die Behauptung zu zeigen, müssen folgende Aussagen
                    gelten.
                    \begin{itemize}
                        \item $F$ ist monoton steigend.
                        \item $F$ ist rechtsseitig stetig.
                        \item $F(-\infty) = 0$ und $F(\infty) = 1$.
                    \end{itemize}

                    Da $F$ auf $(1, 3]$ eine lineare Funktion mit positiver
                    Steigung und sonst eine konstante Funktion ist, ist $F$
                    monoton steigend.

                    Da $F$ aus einer linearen und zwei konstanten Funktionen
                    zusammengesetzt ist, gilt es, an den Schnittstellen $x = 1$
                    und $x = 3$ auf rechtsseitige Stetigkeit zu prüfen.
                    \begin{align*}
                        \begin{split}
                            \lim_{x \searrow 1} F(x)
                            &= \lim_{x \searrow 1}
                            ( \num{0,5} \cdot x - \num{0,5})
                            = \num{0,5} - \num{0,5}
                            = 0
                            = F(1)
                        \end{split} \\
                        \begin{split}
                            \lim_{x \searrow 3} F(x)
                            &= \lim_{x \searrow 1} (1)
                            = 1
                            = F(3)
                        \end{split}
                    \end{align*}

                    Es ist leicht zu sehen, dass auch die dritte Bedindung
                    erfüllt ist.
                    \begin{align*}
                        \begin{split}
                            F(-\infty) = \lim_{x \to -\infty} F(x)
                                = \lim_{x \to -\infty} 0
                                = 0
                        \end{split} \\
                        \begin{split}
                            F(-\infty) = \lim_{x \to -\infty} F(x)
                                = \lim_{x \to -\infty} 1
                                = 1
                        \end{split}
                    \end{align*}
                    $F$ ist also eine Verteilungsfunktion.
                \end{proof}

            \item
                Sei $P$ das zu $F$ gehörige Wahrscheinlichkeitsmaß auf
                $\mathbb{R}$.
                \begin{behaupt}
                Es gilt $P((1, 2]) = \num{0,5}$.
                \end{behaupt}
                \begin{proof}
                    Es gilt $F(x) = P((-\infty, x])$.
                    \begin{equation*}
                        \begin{split}
                            P ((1, 2]) &= P((-\infty, 2]) - P((-\infty, 1]) \\
                            &= F(2) - F(1) \\
                            &= (\num{0,5} \cdot 2 - \num{0,5})
                            - (\num{0,5} \cdot 1 - \num{0,5}) \\
                            &= \num{0,5}
                        \end{split}
                    \end{equation*}
                \end{proof}

            \item
                Eine Dichte von $P$ ist eine Funktion $f$, welche
                \begin{equation}
                    F(x) = P((-\infty, x] = \int_{-\infty}^x f(t) \ \textrm{d}t
                    \label{eq:dichte}
                \end{equation}
                erfüllt.
                Die Ableitung $f$ von $F$ ist eine solche Funktion.
                \begin{equation*}
                    f(x) =
                    \begin{cases}
                        0 & \text{, } x \leq 1 \\
                        \num{0,5} & \text{, } 1 < x \leq 3 \\
                        0 & \text{, } 3 < x \\
                    \end{cases}
                \end{equation*}
                
                Um zu zeigen, dass Gleichung \eqref{eq:dichte} gilt, betrachten
                wir drei Fälle.

                Für $x \leq 1$ gilt
                \begin{equation*}
                    \int_{-\infty}^x f(t)\ \textrm{d}{t}
                    = [0]_{-\infty}^x = 0 = F(x) \text{ .}
                \end{equation*}

                Für $1 < x \leq 3$ gilt
                \begin{equation*}
                    \int_{-\infty}^x f(t)\ \textrm{d}{t}
                    = \int_{-\infty}^1 f(t)\ \textrm{d}{t}
                    + \int_{1}^x f(t)\ \textrm{d}{t}
                    = [\num{0,5}x]_1^x = \num{0,5}x - \num{0,5} = F(x)
                    \text{ .}
                \end{equation*}
                
                Für $3 < x$ gilt
                \begin{equation*}
                    \int_{-\infty}^3 f(t)\ \textrm{d}{t}
                    = \int_3^\infty f(t)\ \textrm{d}{t}
                    + \int_{1}^x f(t)\ \textrm{d}{t}
                    = [\num{0,5}x]_1^3 = 1 = F(x)
                    \text{ .}
                \end{equation*}

                Daher ist $f$ eine Dichte von $P$.

        \end{enumerate}

    \item
        Gegeben sei die Funktion $g$:
        \begin{equation*}
            g(x) =
            \begin{cases}
                0, & x < 0 \\
                \sin(x), & 0 \leq x < \num{1,5} \cdot \pi \\
                0, & \num{1,5} \cdot \pi \leq x
            \end{cases}
        \end{equation*}
        \begin{behaupt}
            $g$ ist keine Dichte.
        \end{behaupt}
        \begin{proof}
            Eine Dichte ist eine Funktion, welche $\mathbb{R}$ auf
            \emph{nicht negative}, reelle Zahlen abbildet und deren Integral
            über $\mathbb{R}$ gleich 1 ist.
            Da $g$ im Intervall $[0,\ \num{1,5} \cdot \pi)$ der Sinusfunktion
            entspricht und diese auf dem Teilstück $(\pi,\ \num{1,5} \cdot \pi)$
            negative Werte annimmt, kann $g$ keine Dichte sein.
        \end{proof}

\end{enumerate}


\end{document}
